\documentclass{article}
\usepackage[utf8]{inputenc}
\usepackage[english]{babel}
\usepackage[]{amsthm} 
\usepackage{enumitem}
\usepackage{array}
\usepackage{amsmath}
\usepackage[]{amssymb} 
\usepackage{gensymb}
\DeclareMathOperator{\lcm}{lcm}

\title{HW 2 - MATH403}
\author{Danesh Sivakumar}
\date\today


\begin{document}
\maketitle 


\subsection*{Problem 1 (Chapter 3, Exercise 20)}
For any group elements $a$ and $b$, prove that $|ab| = |ba|$.
\begin{proof}

Suppose that $|ab| = n$, so that $(ab)^n = e$. Then:
\[ (ab)^n = ababab \cdots ab = ababab \cdots abaa^{-1} = a(bababa \cdots ba)a^{-1} = a(ba)^na^{-1} = e\]
Right multiplying both sides of the last equality by $a$ yields $a(ba)^n = a$, which implies that $(ba)^n = e$. This means that $|ba|$ divides $n$, or that $|ba|$ divides $|ab|$, so that $|ba| \leq |ab|$.

Now suppose that $|ba| = m$, so that $(ba)^m = e$. Then:
\[ (ba)^m = bababa \cdots ba = bababa \cdots babb^{-1} = b(ababab \cdots ab)b{^-1} = b(ab)^mb^{-1} = e\]
Right multiplying both sides of the last equality by $b$ yields $b(ab)^m = b$, which implies that $(ab)^m = e$. This means that $|ab|$ divides $m$, or that $|ab|$ divides $|ba|$, so that $|ab| \leq |ba|$.

Because $|ab| \leq |ba|$ and $|ba| \leq |ab|$, it follows that $|ab| = |ba|$.

\end{proof}


\subsection*{Problem 2 (Chapter 3, Exercise 28)}
Prove that a group with two elements of order 2 that commute must have a subgroup of order 4.
\begin{proof}

Let $a, b \in G$ with $|a| = 2$, $|b| = 2$, and $ab = ba$. Consider $H = \{e, a, b, ab\} \subseteq G$. We claim that $H$ is a subgroup of $G$. We must show that (1) $x, y \in H \implies x \ast y \in H$ and (2) $x \in H \implies x^{-1} \in H$. Clearly $H$ is nonempty; to this end, observe the Cayley table of $H$: 

\[
    \setlength{\extrarowheight}{3pt}
    \begin{array}{l|*{4}{l}}
        & e   & a   & b  & ab  \\
    \hline
    e   & e   & a   & b  & ab  \\
    a   & a   & a^2   & ab & a^2b  \\
    b   & b   & ba  & b^2  & bab  \\
    ab  & ab  & aba   & abb  & abab  \\
    \end{array} 
\]

Using the fact that $|a| = 2$ and $|b| = 2$, we deduce that $a^2 = e$ and $b^2 = e$. Furthermore, because $a$ and $b$ commute, observe that $aba = aab = b$ and $bab = bba = a$. Also, note that $abab = aabb = e$. With this, the simplified Cayley table becomes:

\[
    \setlength{\extrarowheight}{3pt}
    \begin{array}{l|*{4}{l}}
        & e   & a   & b  & ab  \\
    \hline
    e   & e   & a   & b  & ab  \\
    a   & a   & e   & ab & b  \\
    b   & b   & ab  & e  & a  \\
    ab  & ab  & b   & a  & e  \\
    \end{array} 
\]

Since each row and column of the Cayley table contains each element exactly once, $H$ is closed, so (1) is satisfied. $e^{-1} = e$ trivially, and because $a^2 = e$ and $b^2 = e$ it follows that $b^{-1} = b$ and $a^{-1} = a$. Because $abab = (ab)^2 = e$, it follows that $(ab)^{-1} = ab$; thus, each element in $H$ has an inverse (namely itself), so (2) is satisfied. Thus, $H$ is a subgroup of order 4.

\end{proof}

\subsection*{Problem 3 (Chapter 3, Exercise 38)}
Let $G$ be an Abelian group and $H = \{x \in G| \text{ }|x| \text{ is odd}\}$. Prove that $H$ is a subgroup of $G$.

\begin{proof}

We must show that (1) $x, y \in H \implies x \ast y \in H$ and (2) $x \in H \implies x^{-1} \in H$. Note that $H$ is nonempty because $|e| = 1$, so $e \in H$. To prove (1), consider $a, b \in H$. Because $|a|, |b|$ are odd, we have that $|a| = 2k+1$ and $|b| = 2l+1$ for nonnegative integers $k, l$. Because $G$ is abelian, it follows that $ab = ba$, so $|ab|$ divides $(2k+1)(2l+1) = 4kl + 2k + 2l + 1 = 2(2kl+k+l) + 1$. By closure of integers under multiplication and addition, $2kl+k+l = c \in \mathbb{N}$, so $|ab|$ divides $2c+1$, which is an odd nonnegative integer. To prove $|ab|$ is odd, suppose not, that is, that $|ab| = 2j$ for some nonnegative integer $j$. Then our previous result implies that there exists $m \in \mathbb{Z}$ such that $2jm = 2c + 1 \implies 2(jm-c) = 1 \implies (jm-c) = \frac{1}{2}$, which is a contradiction because $j$, $c$ and $m$ are all integers; thus $|ab|$ is odd, so $ab \in H$, proving (1). By a result in the previous homework, we have that the order of any element and its inverse are the same, so that for all $a \in H$, we have that $|a| = 2k+1 \implies |a^{-1}|=2k+1$, so $a^{-1} \in H$, proving (2). Thus, $H$ is a subgroup of $G$.


\end{proof}

\subsection*{Problem 4 (Chapter 3, Exercise 42)}
In the group $\mathbb{Z}$, find:
\begin{enumerate}[label=(\alph*)]
\item $\langle 8, 14 \rangle$;
\item $\langle 8, 13 \rangle$;
\item $\langle 6, 15 \rangle$;
\item $\langle m, n \rangle$;
\item $\langle 12, 18, 45 \rangle$;
\end{enumerate}
In each part, find an integer $k$ such that the subgroup is $\langle k \rangle$.
\begin{proof}

Note that from a theorem in class, we have that $\langle m, n \rangle = \langle \gcd{(m, n)}\rangle$, so that: 

\begin{enumerate}[label=(\alph*)]
\item $\langle 8, 14 \rangle$ = $\langle \gcd{(8, 14)} \rangle$ = $\langle 2 \rangle$
\item $\langle 8, 13 \rangle$ = $\langle \gcd{(8, 13)} \rangle$ = $\langle 1 \rangle = \mathbb{Z}$
\item $\langle 6, 15 \rangle$ = $\langle \gcd{(6, 15)} \rangle$ = $\langle 3 \rangle$
\item $\langle m, n \rangle$ = $\langle \gcd{(m, n)}\rangle$
\item $\langle 12, 18, 45 \rangle$ = $\langle \gcd{(12, 18, 45)}\rangle$ = $\langle 3 \rangle$
\end{enumerate}

\end{proof}


\subsection*{Problem 5 (Chapter 3, Exercise 46)}
Suppose $a$ belongs to a group and $|a| = 5$. Prove that $C(a) = C(a^3)$. Find an element $a$ from some group such that $|a| = 6$ and $C(a) \neq C(a^3)$.

\begin{proof}

We must show that (1) $C(a) \subseteq C(a^3)$ and (2) $C(a^3) \subseteq C(a)$. To prove (1), suppose that $b \in C(a)$. Then $ab = ba$, so that
\[a^3b = aaab = aaba = abaa = baaa = ba^3 \]
showing that $b \in C(a^3)$, proving (1). To prove (2), suppose that $b \in C(a^3)$. Then $a^3b = ba^3$; noting that because $|a| = 5 \implies a^5 = e$, observe
\[ ab = a^5ab = a^6b = a^3a^3b = a^3ba^3 = ba^3a^3 = ba^6 = baa^5 = ba\]
showing that $b \in C(a)$, proving (2). Since $C(a) \subseteq C(a^3)$ and $C(a^3) \subseteq C(a)$, it follows that $C(a) = C(a^3)$.

For the counterexample, consider the dihedral group $D_6$, wherein $a \in D_6$ corresponds to a $60 \degree$ rotation, and $b \in D_6$ corresponds to a reflection about the horizontal axis. Observe that $|a| = 6$, and $ba^3 = a^3b$, but $ba \neq ab$, so that $b \in C(a^3)$ but $b \notin C(a)$, showing that the two centralizers are not equal in this case.


\end{proof}


\subsection*{Problem 6 (Chapter 3, Exercise 74)}
If $H$ and $K$ are nontrivial subgroups of the rational numbers under addition, prove that $H \cap K$ is nontrivial.

\begin{proof}

Suppose that $\frac{a}{b} \in H$ and $\frac{c}{d} \in K$ for nonzero integers $a, b, c, d$. Then by closure of rationals under addition, $a \in H$ and $c \in K$. Applying closure under addition once more shows that $ac \in H$ and $ca \in K$. Because the rationals are commutative under multiplication, $ac = ca$, so that $ac \in H \cap K$.

\end{proof}

\subsection*{Problem 7 (Chapter 4, Exercise 2)}

Suppose that $\langle a \rangle$, $\langle b \rangle$, and $\langle c \rangle$ are cyclic groups of orders 6, 8, and 20, respectively. Find all generators of $\langle a \rangle$, $\langle b \rangle$, and $\langle c \rangle$.

\begin{proof}

From a theorem in class, we have that given $|\langle a \rangle| = n$, all generators of $\langle a \rangle$ are of the form $a^k$, where $\gcd{(n, k)} = 1$. From this, we deduce that the generators of $\langle a \rangle$ are $a$ and $a^5$; the generators of $\langle b \rangle$ are $b$, $b^3$, $b^5$, and $b^7$; the generators of $\langle c \rangle$ are $c$, $c^3$, $c^7$, $c^9$, $c^{11}$, $c^{13}$, $c^{17}$, and $c^{19}$.

\end{proof}

\subsection*{Problem 8 (Chapter 4, Exercise 4)}
List the elements of the subgroups $\langle 3 \rangle$ and $\langle 15 \rangle$ in $\mathbb{Z}_{18}$. Let $a$ be a group element of order 18. List the elements of the subgroups $\langle a^3 \rangle$ and $\langle a^{15} \rangle$

\begin{proof}

$\langle 3 \rangle = \{0, 3, 6, 9, 12, 15\}$. \\
Note that in $\mathbb{Z}_{18}$, $15 \equiv -3$, so that $\langle 15 \rangle = \langle -3 \rangle = \langle 3 \rangle = \{0, 3, 6, 9, 12, 15\}$. \\
$\langle a^3 \rangle = \{(a^3)^n\} = a^{3n} \in \langle a \rangle = \{e, a^3, a^6, a^9, a^{12}, a^{15}\}$ \\
$\langle a^{15} \rangle = \langle a^{-3} \rangle = \langle a^{3} \rangle = \{e, a^3, a^6, a^9, a^{12}, a^{15}\}$.

\end{proof}

\subsection*{Problem 9 (Chapter 4, Exercise 8)}
Let $a$ be an element of a group and let $|a| = 15$. Compute the orders of the following elements of $G$.
\begin{enumerate}[label=(\alph*)]
\item $a^3$, $a^6$, $a^9$, $a^{12}$
\item $a^5$, $a^{10}$
\item $a^2$, $a^4$, $a^8$, $a^{14}$
\end{enumerate}

\begin{proof}

From a formula proven in class, we have that if $|a| = n$, then $|a^k| = \frac{n}{\gcd{(n, k)}}$, so that: 

\begin{enumerate}[label=(\alph*)]
\item For all $k \in \{3, 6, 9, 12\}$, $\gcd{(k, 15)} = 3$, so it follows that $|a^k| = \frac{15}{3} = 5$.
\item For all $k \in \{5, 10\}$, $\gcd{(k, 15)} = 5$, so it follows that $|a^k| = \frac{15}{5} = 3$.
\item For all $k \in \{2, 4, 8, 14\}$, $\gcd{(k, 15)} = 1$, so it follows that $|a^k| = \frac{15}{1} = 15$.

\end{enumerate}

\end{proof}

\subsection*{Problem 10 (Chapter 4, Exercise 14)}
Suppose that a cyclic group $G$ has exactly three subgroups: $G$ itself, $\{e\}$, and a subgroup of order 7. What is $|G|$? What can you say if 7 is replaced with $p$ where $p$ is a prime?

\begin{proof}

By the fundamental theorem of cyclic groups, we have that the subgroups of a cyclic group $G$ have orders equal to the divisors of the order of $G$. From this, we know that 7 divides $|G|$. The fact that there are exactly three subgroups means that $|G| = 7 \cdot 7 = 49$, because otherwise $|G|$ would not have three divisors and thus not have three subgroups, contradicting the supposition. More generally, $|G| = p^2$ if $p$ is a prime, and there are three subgroups: one whose order is $p^2$, one whose order is $p$, and one whose order is 1 (the identity).

\end{proof}

\subsection*{Problem 11 (Chapter 4, Exercise 32)}
Determine the subgroup lattice for $\mathbb{Z}_{12}$. Generalize to $\mathbb{Z}_{p^2q}$, where $p$ and $q$ are distinct primes.

\begin{proof}

Note that the proper divisors of 12 are 1, 2, 3, 4, and 6, so we will consider the subgroups generated by these elements:

\[ \langle 1 \rangle = \mathbb{Z}_{12}\]
\[ \langle 2 \rangle = \{0, 2, 4, 6, 8, 10\}\]
\[ \langle 3 \rangle = \{0, 3, 6, 9\}\]
\[ \langle 4 \rangle = \{0, 4, 8\}\]
\[ \langle 6 \rangle = \{0, 6\}\]

To construct the subgroup lattice, we draw connections between any two subgroups whose elements are fully contained in other. \\

For the general case, notice that the proper divisors of $p^2q$ are $p$, $p^2$, $q$, $pq$ and 1, so that:

\[ \langle 1 \rangle = \mathbb{Z}_{p^2q}\]
\[ \langle p \rangle = \{0, p, 2p, \cdots, p^2, \cdots, p^2q\}\]
\[ \langle q \rangle = \{0, q, 2q, \cdots, p^2q\}\]
\[ \langle pq \rangle = \{0, pq, 2pq, \cdots, p^2q\}\]
\[ \langle p^2 \rangle = \{0, p^2, \cdots, p^2q\}\]


\end{proof}

\subsection*{Problem 12 (Chapter 4, Exercise 44)}
Which of the following numbers could be the exact number of elements of order 21 in a group: 21600, 21602, 21604?

\begin{proof}

Using the fact that in any finite group, the number of elements of order $d$ is a multiple of $\Phi{(d)}$, we deduce that the number of elements of order 21 is a multiple of $\Phi{(21)} = \Phi{(3)}\Phi{(7)} = (3-1)(7-1) = 2 \cdot 6 = 12$. The only number that is a multiple of 12 is 21600, so the only possible choice is 21600.

\end{proof}

\subsection*{Problem A}
Prove that every finite subgroup of $\left(\mathbb{C}^{\ast},\times\right)$ is cyclic.

\begin{proof}

Let $H \in \left(\mathbb{C}^{\ast},\times\right)$ be a finite subgroup. We claim that $H$ is comprised of $n$th roots of unity. To this end, suppose not; that is, that $|a| \in H \neq 1$, where $|a|$ denotes the magnitude of a. There are two cases: (1) $|a| > 1$ and (2) $|a| < 1$. Let $a = re^{i\vartheta}$ where $r \neq 1$. For (1), we have that $|a^2| = |r^2e^{i2\vartheta}| = r^2 > r =  |re^{i\vartheta}| = |a|$, so that $|a^2| > |a|$. Suppose that $|a^{k+1}| > |a^k|$. Then $|a^{k+2}| = |a^{k+1}a| = |a^{k+1}||a| > |a^{k+1}|$, so that for all $n \in \mathbb{N}$ it follows that $|a^{n+1}| > |a^n|$, so that $a^{n+1} \neq a^n$, contradicting the fact that $H$ is finite. For (2), we have that $|a^2| = |r^2e^{i2\vartheta}| = r^2 < r =  |re^{i\vartheta}| = |a|$, so that $|a^2| < |a|$. Suppose that $|a^{k+1}| < |a^k|$. Then $|a^{k+2}| = |a^{k+1}a| = |a^{k+1}||a| < |a^{k+1}|$, so that for all $n \in \mathbb{N}$ it follows that $|a^{n+1}| < |a^n|$, so that $a^{n+1} \neq a^n$, contradicting the fact that $H$ is finite. Thus, $|a| = 1$, so that $H$ can only be a group of $n$th roots of unity whose elements are of the form $e^{\frac{2k\pi\i}{n}}$. Letting $k = 1$ gives us an $a \in H$ such that $\langle a \rangle = H$, so that $a = e^{\frac{2\pi\i}{n}}$ is a generator for $H$, proving that $H$ is cyclic.

\end{proof}

\subsection*{Problem B}
Show that the subgroup $\langle a,b\rangle$ is cyclic for any $a,b\in\left(\mathbb{Q},\:+\right)$.

\begin{proof}

Let $a = \frac{m}{n}$ and $b = \frac{p}{q}$, where $m, n, p, q \in \mathbb{Z}$. Define $\gcd{(a, b)} = \frac{\gcd{(m, p)}}{\lcm{(n, q)}}$. First, we prove $\langle a, b \rangle \subseteq \langle \gcd{(a, b)} \rangle$: let $c \in \langle a, b \rangle$, so that $c = xa+yb$. But $\gcd{(a, b)}$ divides both $a$ and $b$ by definition, so $a=l\gcd{(a, b)}$ and $b=k\gcd{(a, b)}$ for nonnegative integers $k$ and $l$. This implies that $c = \gcd{(a, b)}(xl + yk)$ by substitution, so that $c \in \langle \gcd{(a, b)} \rangle$. Now we show $\langle \gcd{(a, b)} \rangle \subseteq \langle a, b \rangle$: let $c \in \gcd{(a, b)}$, so that $c = k\gcd{(a, b)}$ for some $k \in \mathbb{N}$. By Bezout's theorem, we have that there exist $x, y \in \mathbb{Z}$ such that $ax + by = \gcd{(a, b)}$. Thus it follows that $c = k(ax + by) = kax + kby$. Because $kx, ky \in \mathbb{Z}$, it follows that $c \in \langle a, b \rangle$. Thus $\langle a, b \rangle \subseteq \langle \gcd{(a, b)} \rangle$ and $\langle \gcd{(a, b)} \rangle \subseteq \langle a, b \rangle$, so that $\langle \gcd{(a, b)} \rangle = \langle a, b \rangle$, meaning $\langle a, b \rangle$ is generated by $\gcd{(a, b)}$ and thus cyclic, so the result for integers holds more generally for rationals.

\end{proof}



\end{document}
