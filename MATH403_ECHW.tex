\documentclass{article}
\usepackage[utf8]{inputenc}
\usepackage[english]{babel}
\usepackage[]{amsthm} 
\usepackage{enumitem}
\usepackage{array}
\usepackage{amsmath}
\usepackage{amssymb}
\usepackage{textcomp}
\usepackage{gensymb}
\DeclareMathOperator{\lcm}{lcm}
\newif\ifanswers
\answerstrue

\title{EC HW - MATH403}
\author{Danesh Sivakumar}
\date{May 12th, 2022}


\begin{document}
\maketitle


The first six problems (7 points each) may be true or false. Give a brief reason why.
\begin{enumerate}
\item If $K$ is a field and $L \subseteq K$ is a subring, then $L$ is an integral domain.
\ifanswers
\textbf{Solution:}
True; we will show that $L$ has no zero divisors. Suppose $x, y \in L$ with $xy = 0$. Then it
follows that $x, y \in K$, meaning that $x = 0$ or $y = 0$ because $K$ is a field.
\fi
\item If $\sigma = (123)(45) \in S_5$ and $\tau = (1, 2)(3, 4, 5) \in S_5$ then there exists
$\lambda \in S_5$ with $\lambda\sigma\lambda^{-1} = \tau$
\newline
\ifanswers
\textbf{Solution:}
True; both $\sigma$ and $\tau$ have the same cycle structure (they are each the product of a
2-cycle and 3-cycle), so they are conjugate.
\fi

\item There exists a finite field with 24 elements.
\newline
\ifanswers
\textbf{Solution:}
False; the order of any finite field must always be the power of a prime $p$, and 24 is not 
a power of a prime.
\fi

\item Every group of order 7 is isomorphic to the multiplicative subgroup of the group of 
nonzero elements of a finite field.
\newline
\ifanswers
\textbf{Solution:}
True; every group of order 7 is the same up to isomorphism because 7 is prime, so take any finite field with 7 elements (which is guaranteed to exist because 7 is prime). 
\fi

\item If $G$ is a group of order 245, then $G$ contains a normal subgroup isomorphic to 
$\mathbb{Z}_5$
\newline
\ifanswers
\textbf{Solution:}
True; observe that $245 = 7^2 \cdot 5$, meaning that the number of $5$-Sylow subgroups of $G$ 
is equal to $1 \mod {5}$ and divides 49; this implies that there is only one $5$-Sylow subgroup,
so it is normal due to its uniqeueness. Furthermore, this subgroup has prime order 5, meaning 
that it must be cyclic and thus isomorphic to $\mathbb{Z}_5$.
\fi

\item If $H$ is a subgroup of $G$ with $|G| = 180$ and the index of $H$ in $G$ is 12, then
$H$ contains an element of order 3 but no element of order 2.
\newline
\ifanswers
\textbf{Solution:}
True; observe that $|H| = 180/12 = 15$, and 2 does not divide 15 so there cannot be an element
of order 2. However, 3 (a prime) divides 15, and by Cauchy's theorem it follows that there
is an element of order 3.
\fi

\item Up to isomorphism, there exist exactly two non-isomorphic abelian groups of order 20
called $A$ and $B$. Which of the following abelian groups of order 20 are isomorphic to $A$
and which are isomorphic to $B$? a) $\mathbb{Z}_{20}$; b) $\mathbb{Z}_{10} \times 
\mathbb{Z}_2$; c) $\mathbb{Z}_5 \times \mathbb{Z}_4$; d) $U_{25}$ the units in the ring 
$\mathbb{Z}_{25}$; e) the subgroup of $S_9$ generated by $a = (12345)$, $b = (67)$ and 
$c = (89)$; f) the kernel of the homomorphism $f \colon \mathbb{Z}_{20} \times 
\mathbb{Z}_2 \to \mathbb{Z}_2$ given by $f(a \mod{(20)}, b \mod{(2)}) = (a + b) \mod{(2)}$.
\newline
\ifanswers
\textbf{Solution:}
Suppose $A$ is cyclic and $B$ is not cyclic. Then $A$ is isomorphic to a) [by definition], c)
[because 4 and 5 are relatively prime], and d) [because 25 is the power of an odd prime]. Next, 
$B$ is isomorphic to b) [because 10 and 2 are not relatively prime], e) 
[because this subgroup is isomorphic to $\mathbb{Z}_5 \times \mathbb{Z}_2 \times 
\mathbb{Z}_2$, which is not cyclic] and f) [because the kernel is all tuples $(x, y)$ such 
that $x$ and $y$ have the same parity in their respective groups, which is isomorphic to 
$\mathbb{Z}_{10} \times \mathbb{Z}_2$].
\fi

\item Let $f = x^3 + x + 1 \in \mathbb{Z}_3[x]$.
\begin{enumerate}[label=(\arabic*)]
	\item Determine the unique factorization of $f$ in $\mathbb{Z}_3[x]$.
	\item Show $\mathbb{Z}_3[x]/(f)$ is isomorphic as rings to the product $\mathbb{Z}_3
	\times F$ where $F$ is a field of 9 elements.
\end{enumerate}
\ifanswers
\textbf{Solution:}

\begin{enumerate}[label=(\arabic*)]
	\item Observe that 1 is a root of $f$, meaning that $(x-1)$ is a factor. We know that the
	final answer will have the form $(x-1)(ax^2 + bx + c)$; expanding and matching coefficients
	yields $(x-1)(x^2 + x - 1)$.	
	\item Observe that $(x-1)$ and $(x^2 + x - 1)$ are irreducible polynomials in 
	$\mathbb{Z}_3[x]$, meaning that the ideals generated by them are comaximal. Thus, by the
	Chinese Remainder Theorem there exists an isomorphism $\phi \colon \mathbb{Z}_3[x]/(f) \to 
	\mathbb{Z}_3[x]/(x-1) \times \mathbb{Z}_3[x]/(x^2 + x - 1)$. Now observe that the coset 
	representatives in the first component will be constant polynomials $a$, because all higher
	powers will be absorbed by the coset. This gives 3 choices for $a$, meaning that the
	first component is isomorphic to a group of order 3, and $\mathbb{Z}_3$ is the only group
	of order 3 up to isomorphism. The coset representatives in the second component will have
	the form $ax + b$, because all powers higher than $1$ will be absorbed by the coset. Thus,
	we have $3 \cdot 3 = 9$ choices for the coefficients, and thus this set will have size 9. 
	But the second component is also a field, because the coset is a maximal ideal; thus we have 	that the factor ring is isomorphic to $\mathbb{Z}_3 \times F_9$.
\end{enumerate}
\fi

\item
\begin{enumerate}[label = \alph*)]
	\item Let $\mathbb{F}$ be a field and $I \neq \mathbb{F}[x]$ an ideal in $\mathbb{F}[x]$.
	Suppose $I$ contains an irreducible polynomial $f$. Show 
	$I = (f) = \{gf \mid g \in F[x]\}$
	\item Use the automorphism $\varphi \colon \mathbb{Q}[x] \to \mathbb{Q}[x]$ given by 
	$\varphi(x) = x + 1$ to show $p = x^4 + 1$ is irreducible in $\mathbb{Q}[x]$.
	\item Let $\alpha = \sqrt{i} \in \mathbb{C}$ and $\mathbb{Q}[\alpha] = \{p(\alpha) \mid
	p \in \mathbb{Q}[x]\} \subseteq \mathbb{C}$. $\mathbb{Q}[\alpha]$ clearly is a subring
	of $\mathbb{C}$. Use the evaluation map $ev_{\alpha} \colon \mathbb{Q}[x] \to 
	\mathbb{Q}[\alpha]$ to show $\mathbb{Q}[\alpha]$ is a subfield of $\mathbb{C}$.
\end{enumerate}
\ifanswers
\textbf{Solution:}
\begin{enumerate}[label=\alph*)]
	\item Since $I$ contains $f$, it follows that $(f) \subseteq I$. Now because $f$ is 
	irreducible over $\mathbb{F}[x]$, it follows that $(f)$ is a maximal ideal, call it $J$.
	By the maximality of $J$, we have for any ideal $K$ containing $J$ that $K = J$ or
	$K = \mathbb{F}[x]$. Note that $I$ is an ideal containing $J$ but $I \neq \mathbb{F}[x]$;
	this means that $I = J = (f)$.
	\item Observe that under the automorphism, we have that 
	\[ p(x+1) = (x+1)^4 + 1 = x^4 + 4x^3 + 6x^2 + 4x + 2\]
	Now by Eisenstein's Criterion, observe that the prime $2$ does not divide $1$, 
	but $2$ divides every other coefficient and $4$ does not divide the constant term; thus we
	conclude that $p(x+1)$ is irreducible in $\mathbb{Q}[x]$ and thus $p(x)$ is irreducible in 
	$\mathbb{Q}[x]$.
	\item Consider the kernel of the evaluation map: we have that Ker $ev_{\alpha} = \{p \in 
	\mathbb{Q}[x] \mid p(\alpha) = 0\}$. Observe that Ker $ev_{\alpha}$ contains $x^4 + 1$.
	As shown previously, this is an irreducible polynomial over $\mathbb{Q}$ and is thus 
	a maximal ideal of $\mathbb{Q}[x]$. Thus, it follows that $\mathbb{Q}[x]/\text
	{Ker $ev_{\alpha}$}$ is a field, but by the First Isomorphism Theorem we have that 
	this is isomorpic to $\mathbb{Q}[\alpha]$; since this is a subring of $\mathbb{C}$ that is
	also a field, it follows that it is a subfield.
\end{enumerate}
\fi


\item
\begin{enumerate}[label = \alph*)]
	\item Let $G$ be a group and $N$ a normal subgroup of $G$. Show $G/N$ is an abelian group
	is and only if for all $g, h \in G, ghg^{-1}h^{-1} \in N$.
	\item Show the alternating group $A_n$ for $n \geq 5$ has no subgroups $H$ of index 
	strictly less than 5.
\end{enumerate}
\ifanswers
\textbf{Solution:}
\begin{enumerate}[label = \alph*)]
	\item Suppose $G/N$ is abelian. Then we have that $(gN)(hN) = (hN)(gN)$, which means that
	$ghN = hgN$, meaning that $ghg^{-1}h^{-1}N = N$, so that $ghg^{-1}h^{-1} \in N$. Now 
	suppose that $ghg^{-1}h^{-1} \in N$. Then we have that $ghg^{-1}h^{-1}N = N$, meaning that
	$ghN = hgN$, so that $(gN)(hN) = (hN)(gN)$, meaning that $G/N$ is abelian.
	\item Consider the homomorphism $\varphi \colon A_n \to S(A_n/H)$ wherein the actions of
	$A_n$ are applied onto the cosets. We have that the right hand side is isomorphic to 
	$S_m$, where $m$ is the index of $H$ in $A_n$. By the First Isomorphism Theorem, we have
	that $A_n/\text{Ker $\varphi$}$ is isomorphic to a subgroup of $S_m$. However, observe
	also that Ker $\varphi$ is a normal subgroup of $A_n$, and the fact that $A_n$ is simple
	means that Ker $\varphi$ is either $A_n$ or trivial. It cannot be $A_n$; if it were,
	taking $x \notin H$ would make $xH \neq H$ so that $\phi(x)$ is not the identity, which
	contradicts the definition of the homomorphism. Thus Ker $\varphi$ is trivial and so 
	the factor group has order at least $n!/2 \geq 5!/2 = 60$. But $S_m$ has order strictly 
	less than $60$, and there is no way that the factor group of order greater than or equal to
	60 could be isomorphic to a group of order strictly less than 60. This is a contradiction,
	meaning that no such subgroups of index strictly less than 5 exist.
\end{enumerate}

\fi

\end{enumerate}

\end{document}
