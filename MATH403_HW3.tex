\documentclass{article}
\usepackage[utf8]{inputenc}
\usepackage[english]{babel}
\usepackage[]{amsthm} 
\usepackage{enumitem}
\usepackage{array}
\usepackage{amsmath}
\usepackage[]{amssymb} 
\usepackage{gensymb}
\DeclareMathOperator{\lcm}{lcm}

\title{HW 3 - MATH403}
\author{Danesh Sivakumar}
\date\today


\begin{document}
\maketitle 


\subsection*{Problem 1 (Chapter 5, Exercise 1)}
Let 
\[\alpha = \begin{bmatrix}
1 & 2 & 3 & 4 & 5 & 6 \\
2 & 1 & 3 & 5 & 4 & 6
\end{bmatrix} \text{\quad and \quad} \beta = \begin{bmatrix}
1 & 2 & 3 & 4 & 5 & 6 \\
6 & 1 & 2 & 4 & 3 & 5
\end{bmatrix}\]
Compute each of the following

\begin{enumerate}[label=\textbf{\alph*.}]
    \item $\alpha^{-1}$
    \item $\beta\alpha$
    \item $\alpha\beta$
\end{enumerate}
\begin{proof}
\begin{enumerate}[label=\textbf{\alph*.}]
    \item \[ \alpha^{-1} = \begin{bmatrix}
1 & 2 & 3 & 4 & 5 & 6 \\
2 & 1 & 3 & 5 & 4 & 6
\end{bmatrix}\]
    \item \[ \beta\alpha = \begin{bmatrix}
1 & 2 & 3 & 4 & 5 & 6 \\
1 & 6 & 2 & 3 & 4 & 5
\end{bmatrix}\]
    \item \[ \alpha\beta = \begin{bmatrix}
1 & 2 & 3 & 4 & 5 & 6 \\
6 & 2 & 1 & 5 & 3 & 4
\end{bmatrix}\]
\end{enumerate}
\end{proof}


\subsection*{Problem 2 (Chapter 5, Exercise 2)}
Let 
\[\alpha = \begin{bmatrix}
1 & 2 & 3 & 4 & 5 & 6 & 7 & 8 \\
2 & 3 & 4 & 5 & 1 & 7 & 8 & 6
\end{bmatrix} \text{\quad and \quad} \beta = \begin{bmatrix}
1 & 2 & 3 & 4 & 5 & 6 & 7 & 8 \\
1 & 3 & 8 & 7 & 6 & 5 & 2 & 4
\end{bmatrix}\]

Write $\alpha$, $\beta$ and $\alpha\beta$ as
\begin{enumerate}[label=\textbf{\alph*.}]
    \item products of disjoint cycles;
    \item products of 2-cycles.
\end{enumerate}


\begin{proof}
Note that
\[ \alpha\beta = \begin{bmatrix}
1 & 2 & 3 & 4 & 5 & 6 & 7 & 8 \\
2 & 4 & 6 & 8 & 7 & 1 & 3 & 5
\end{bmatrix}\]
Then for $\alpha$ we have:
\begin{enumerate}[label=\textbf{\alph*.}]
    \item $\alpha = (12345)(678)$
    \item $\alpha = (15)(14)(13)(12)(68)(67)$
\end{enumerate}
For $\beta$ we have:
\begin{enumerate}[label=\textbf{\alph*.}]
    \item $\beta = (1)(23847)(56)$
    \item $\beta = (27)(24)(28)(23)(56)$
\end{enumerate}
For $\alpha\beta$ we have:
\begin{enumerate}[label=\textbf{\alph*.}]
    \item $\alpha\beta = (12485736)$
    \item $\alpha\beta = (16)(13)(17)(15)(18)(14)(12))$
\end{enumerate}
\end{proof}

\subsection*{Problem 3 (Chapter 5, Exercise 10)}
Show that $A_8$ contains an element of order 15.

\begin{proof}
Consider the permutation $\sigma = (12345)(678)$. Writing $\sigma$ as a product of transpositions shows that $\sigma = (15)(14)(13)(12)(68)(67)$, which is a product of an even number of transpositions; thus $\sigma \in A_8$. Note that because $(12345)$ and $(678)$ are disjoint cycles whose product is $\sigma$, it follows that the order of $\sigma$ is the least common multiple of the orders of its disjoint cycles; the order of each n-cycle is n, so $|(12345)| = 5$ and $|(678)| = 3$, so that $|\sigma| = \lcm{(5, 3)} = 15$. Thus, $\sigma$ is an element of $A_8$ whose order is 15.

\end{proof}

\subsection*{Problem 4 (Chapter 5, Exercise 14)}
Suppose that $\alpha$ is a 6-cycle and $\beta$ is a 5-cycle. Determine whether $\alpha^5\beta^4\alpha^{-1}\beta^{-3}\alpha^5$ is even or odd. Show your reasoning.

\begin{proof}

Let $\alpha = (a_1a_2a_3a_4a_5a_6)$ and $\beta = (b_1b_2b_3b_4b_5)$, with $\alpha$ and $\beta$ not necessarily disjoint. Writing $\alpha$ as a product of transpositions yields \\ $\alpha = (a_1a_6)(a_1a_5)(a_1a_4)(a_1a_3)(a_1a_2)$ and $\beta = (b_1b_5)(b_1b_4)(b_1b_3)(b_1b_2)$, so that $\alpha$ is the product of 5 transpositions and $\beta$ is the product of 4 transpositions. Using the exponents of the expression, we deduce that it can be written as the product of $5(5) + 4(4) + 1(5) + 3(4) + 5(5) = 83$ transpositions, which is odd. Note that if $r$ transpositions are "redundant" (i.e. can be replaced with $\varepsilon$), it follows that $r$ is even, so the parity does not change.

\end{proof}


\subsection*{Problem 5 (Chapter 5, Exercise 26)}

Let $\alpha$ and $\beta$ belong to $S_n$. Prove that $\alpha^{-1}\beta^{-1}\alpha\beta$ is an even permutation.

\begin{proof}
It is helpful to establish that an element of $S_n$ and its inverse have the same parity. This is because for any $\sigma \in S_n$, it follows that $\sigma\sigma^{-1} = \varepsilon$, and because $\varepsilon$ is always the product of an even number of transpositions, it follows that $\sigma$ and $\sigma^{-1}$ both have the same parity. Also, note that the product of two transpositions with the same parity is even, and the product of two transpositions with different parity is odd; this follows from the parity of sums of pairs of integers. We will use the notation $\sigma$ = (parity) to denote the parity of a permutation. We break this down to four cases:

\begin{enumerate}
    \item Suppose $\alpha$ is even and $\beta$ is even. Then it follows that $\alpha^{-1}$ is even and $\beta^{-1}$ is even, so that $\alpha^{-1}\beta^{-1}\alpha\beta$ = (even)(even)(even)(even) = (even)(even)(even) = (even)(even) = (even).
    \item Suppose $\alpha$ is even and $\beta$ is odd. Then it follows that $\alpha^{-1}$ is even and $\beta^{-1}$ is odd, so that $\alpha^{-1}\beta^{-1}\alpha\beta$ = (even)(odd)(even)(odd) = (even)(odd)(odd) = (even)(even) = (even).
    \item Suppose $\alpha$ is odd and $\beta$ is odd. Then it follows that $\alpha^{-1}$ is odd and $\beta^{-1}$ is odd, so that $\alpha^{-1}\beta^{-1}\alpha\beta$ = (odd)(odd)(odd)(odd) = (odd)(odd)(even) = (odd)(odd) = (even).
    \item Suppose $\alpha$ is odd and $\beta$ is even. Then it follows that $\alpha^{-1}$ is odd and $\beta^{-1}$ is even, so that $\alpha^{-1}\beta^{-1}\alpha\beta$ = (odd)(even)(odd)(even) = (odd)(even)(odd) = (odd)(odd) = (even).
\end{enumerate}
The parity of $\alpha^{-1}\beta^{-1}\alpha\beta$ is even in every case, so it is always an even permutation.

\end{proof}


\subsection*{Problem 6 (Chapter 5, Exercise 34)}
If $\alpha$ and $\beta$ are distinct 2-cycles, what are the possibilities for $|\alpha\beta|$?

\begin{proof}
If $\alpha$ and $\beta$ are distinct 2-cycles, there are two cases: 
\begin{enumerate}
    \item $\alpha$ and $\beta$ both move the same element (i.e. they are not disjoint), so that WLOG $\alpha = (ac)$ and $\beta = (ab)$. This implies that $\alpha\beta = (ac)(ab) = (abc)$, so that $|\alpha\beta| = 3$. Note that $(ab) = (ba)$ and $b, c$ are arbitrary, so that generality holds.
    \item $\alpha$ and $\beta$ both move distinct elements (i.e. they are disjoint), so that $\alpha = (ab)$ and $\beta = (cd)$. This implies that $\alpha\beta = (ab)(cd)$; because $\alpha\beta$ is the product of two disjoint 2-cycles, it follows that $|\alpha\beta| = \lcm{(2, 2)} = 2$
\end{enumerate}
Thus $|\alpha\beta| = 3$ or $|\alpha\beta| = 2$.

\end{proof}

\subsection*{Problem 7 (Chapter 5, Exercise 46)}

Show that in $S_7$, the equation $x^2 = (1234)$ has no solutions but the equation $x^3 = (1234)$ has at least two.

\begin{proof}

Note that $(1234) = (14)(13)(12)$, so that it is an odd permutation. But $x^2$ is an even permutation regardless of the parity of $x$, and a permutation can never be both even and odd; thus (even) = $x^2 = (1234) = $ (odd) is impossible and thus has no solutions. \\
However, $x = (4321)$ satisfies $x^3 = (1234)$; this is because $x^{-1} = x^3$, since the order of $x$ is 4, meaning $x^4 = \varepsilon$. Note that because disjoint cycles commute and the order of any 3-cycle is 3, it follows that $x = (4321)(567)$ also satisfies the equation; we have that $x^3 = (4321)(4321)(4321)(567)(567)(567) = (1234)\varepsilon = (1234)$. Thus, the equation $x^3 = (1234)$ has at least two solutions in $S_7$. 

\end{proof}

\subsection*{Problem 8 (Chapter 5, Exercise 53)}
Show that $A_5$ has 24 elements of order 5, 20 elements of order 3, and 15 elements of order 2.

\begin{proof}
For the elements of order 5, note that the only form is $\sigma = (a_1a_2a_3a_4a_5)$; this has $\frac{5\cdot4\cdot3\cdot3\cdot2\cdot1}{5} = 24$ possibilities. \\
For the elements of order 3, note that the only form is $\sigma = (a_1a_2a_3)$, which has $\frac{5\cdot4\cdot3}{3} = 20$ possibilities. \\
For the elements of order 2, note that the only form is $\sigma = (a_1a_2)(a_3a_4)$; $\sigma = (a_1a_2)$ does not work because it is not in $A_5$. We have $\frac{5\cdot4\cdot3\cdot2}{2\cdot2\cdot2} = 15$ possibilities.
\end{proof}

\subsection*{Problem 9 (Chapter 5, Exercise 58)}

Show that for $n \geq 3$, $Z(S_n) = \{\varepsilon\}$.

\begin{proof}
We want to show that the only element of $S_n$ that commutes with every element in $S_n$ is $\varepsilon$; that is, for any non-identity element $\sigma \in S_n$, it follows that $\tau\sigma \neq \sigma\tau$ for some $\tau \in S_n$. Since the center always contains the identity, it suffices to show that any non-identity element cannot be in the center. We proceed by contradiction; suppose that there exists $\sigma \in Z(S_n), \sigma \neq \varepsilon$. Take distinct $i, j \in \{1,\cdots,n\}$ such that $\sigma(i) = j$. Now construct $\tau$ with the following property: take $k \in \{1,\cdots,n\}$ with $k \neq i, j$ such that $\tau(j) = k$ and $\tau$ fixes $i$; the existence of such a $k$ follows from the stipulation that $n \geq 3$. It follows that $(\sigma\tau)(i) = \sigma(\tau(i)) = \sigma(i) = j$ and $(\tau\sigma)(i) = \tau(\sigma(i)) = \tau(j) = k$. Since $j$ and $k$ were chosen to be distinct, we have that $\sigma\tau \neq \tau\sigma$ for this choice of $\tau$, contradicting the supposition that a non-identity $\sigma$ existed in the center of $S_n$; thus $Z(S_n) = \{\varepsilon\}$.

\end{proof}

\subsection*{Problem 10 (Chapter 5, Exercise 64)}

Find five subgroups of $S_5$ of order 24.

\begin{proof}
We construct each subgroup by fixing an element, and permuting each of the other elements; this works because the order of $S_4$ is 24. To this end, let $H_n$ be the subgroup obtained by fixing $n$ and permuting all other elements with the actions of $S_4$:
\begin{enumerate}
    \item $H_1 = \text{actions of $S_4$ on the set } \{2, 3, 4, 5\}$
    \item $H_2 = \text{actions of $S_4$ on the set } \{1, 3, 4, 5\}$
    \item $H_3 = \text{actions of $S_4$ on the set } \{1, 2, 4, 5\}$
    \item $H_4 = \text{actions of $S_4$ on the set } \{1, 2, 3, 5\}$
    \item $H_5 = \text{actions of $S_4$ on the set } \{1, 2, 3, 4\}$
\end{enumerate}
These are each indeed subgroups, because they are closed and contain inverses by virtue of $S_4$  being a group, and each $H_n \subseteq S_5$.

\end{proof}

\end{document}
