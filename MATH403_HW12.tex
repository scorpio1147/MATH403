\documentclass{article}
\usepackage[utf8]{inputenc}
\usepackage[english]{babel}
\usepackage[]{amsthm} 
\usepackage{enumitem}
\usepackage{array}
\usepackage{amsmath}
\usepackage{textcomp}
\usepackage[]{amssymb} 
\usepackage{gensymb}
\DeclareMathOperator{\lcm}{lcm}

\title{HW 12 - MATH403}
\author{Danesh Sivakumar}
\date{May 6th, 2022}


\begin{document}
\maketitle 

\subsection*{Problem 1}
Let $G = \langle a \rangle$ be a group of order 46. Classify the elements according
to their order starting with the identity. Bunch the elements of the same order 
together.
\begin{proof}
	We proceed by using the formula 
	\[ \big|a^k\big| = \frac{n}{\gcd{(n, k)}}\]
	given that $|a| = n = 46$ and $1 \leq k \leq 46$. Also, the order of an element
	always divides the order of the group by Lagrange's theorem, so it suffices to 
	consider only the divisors of 46. Furthermore, observe that 46 is the product of
	two primes $23$ and $2$, meaning that $\gcd{(46, k)} = 2$ where 
	$1 \leq k \leq 45$ is even and $\gcd{(46, l)} = 1$ where $1 \leq l \leq 45$ is 
	odd and $l \neq 23$. Thus we have:
	\[ \text{Elements of order 1: } e\]
	\[ \text{Elements of order 2: } a^{23}\]
	\[ \text{Elements of order 23: } a^{2}, a^{4}, a^{6}, a^{8}, a^{10}, a^{12}, a^{
	14}, a^{16}, a^{18}, a^{20}, a^{22}, a^{24}, \]
	\[ a^{26}, a^{28}, a^{30}, a^{32}, a^{34}, a^{36}, a^{38}, a^{40}, a^{42},
	a^{44}\]
	\[ \text{Elements of order 46: } a, a^{3}, a^{5}, a^{7}, a^{9}, a^{11}, a^{
	13}, a^{15}, a^{17}, a^{19}, a^{21}, \]
	\[ a^{25}, a^{27}, a^{29}, a^{31}, a^{33}, a^{35}, a^{37}, a^{39}, a^{41},
	a^{43}, a^{45}\]
	This checks out, because there are $\phi(d)$ elements of order $d$ for every
	positive divisor $d$ of 46, as per the Fundamental Theorem of Cyclic Groups.
\end{proof}

\subsection*{Problem 2}
In any finite group $G$, show that the number of non-identity elements satisfying
the equation $x^5 = e$ is a multiple of 4.
\begin{proof}

	Observe that because $x^5 = e$, it follows that the order of $x$ divides 5. But
	since $x$ is not the identity, it follows that the order of $x$ is 5. By a
	theorem covered in class, we have that the number of elements of order 5 in $G$
	is a multiple of $\phi(5) = 4$. 
	

\end{proof}

\subsection*{Problem 3}
\begin{enumerate}
	\item Find five distinct subgroups of order 12 in $S_5$
	\item Is the subgroup $\langle (12345) \rangle$ of $S_5$ normal? Why or why not?
\end{enumerate}
\begin{proof}
\qquad
\begin{enumerate}
	\item 
	We construct each subgroup by fixing an element, and permuting each of 
	the other elements; this works because the order of $A_4$ is 12. To this 
	end, let $H_n$ be the subgroup obtained by fixing $n$ and permuting all 
	other elements with the actions of $A_4$:
	\begin{enumerate}
		\item $H_1 = \text{actions of $A_4$ on the set } \{2, 3, 4, 5\}$
		\item $H_2 = \text{actions of $A_4$ on the set } \{1, 3, 4, 5\}$
		\item $H_3 = \text{actions of $A_4$ on the set } \{1, 2, 4, 5\}$
		\item $H_4 = \text{actions of $A_4$ on the set } \{1, 2, 3, 5\}$
		\item $H_5 = \text{actions of $A_4$ on the set } \{1, 2, 3, 4\}$
	\end{enumerate}	
	These are each indeed subgroups, because they are closed and 
	contain inverses by virtue of $A_4$  being a group, and each $H_n \subseteq S_5$
	\item 
	No, the subgroup is not normal; observe that 
	\[ \langle (12345) \rangle = \{(12345), (13524), (14253), (15432), \epsilon \}\]
	but $(12)(12345)(12) = (13452) \notin \langle (12345) \rangle$.
\end{enumerate}
\end{proof}

\subsection*{Problem 4}
Prove directly that if $G/Z(G)$ is cyclic, then $G$ is Abelian.
\begin{proof}
	We must show that if $G/Z(G)$ is cyclic, it follows that $Z(G) = G$, meaning that 
	the only element of the factor group $G/Z(G)$ is $Z(G)$. Because $G/Z(G)$ is 
	cyclic, it follows that $G/Z(G) = \langle gZ(g) \rangle$ for some $g \in G$. Now 
	let $a \in G$ be arbitrary. It follows that $aZ(G) = (gZ(G))^i = g^iZ(G)$, which 
	means that $a = g^iz$ for some $z \in Z(G)$. But observe that $z \in C(g)$ and 
	$g^i \in C(g)$, meaning that $a = g^iz \in C(g)$. But since $a$ was arbitrary, 
	this means that every element is in $C(g)$; this means that $g$ commutes with 
	every element in $G$, meaning $g \in Z(G)$. Thus we conclude that $G/Z(G) = Z(G)$
	and thus $G = Z(G)$ so that $G$ is Abelian. 
\end{proof}

\subsection*{Problem 5}
Let $K$ be a $p$-Sylow subgroup of a finite group $G$ for some prime $p$ and let 
$N(K)$ be the normalizer of $K$. If an element $x$ of $N(K)$ has order a power of 
$p$, show that $x$ is in $K$.
\begin{proof}
	Observe that $K$ is normal in $N(K)$. This means that $K$ is the only $p$-Sylow
	subgroup of $N(K)$; this is because $K$ is normal in $N(K)$ and by Sylow's Second
	Theorem, we have that $K$ is the unique $p$-Sylow subgroup of $N(K)$. Now consider
	the subgroup generated by $x$, which has order $p^k$ for some integer $k$; by
	definition, this is a $p$ subgroup of $N(K)$. However, there is only one
	$p$-Sylow subgroup of $N(K)$, which is $K$ itself, meaing that $\langle x \rangle$
	must be contained in $K$; thus $x \in K$ because every $p$ subgroup is contained
	within the $p$-Sylow subgroup. 
\end{proof}

\subsection*{Problem 6}
Let $a$ be an element of order five. Show that $C(a) = C(a^3)$.
\begin{proof}
	We must show that (1) $C(a) \subseteq C(a^3)$ and (2) $C(a^3) \subseteq C(a)$. 
	To prove (1), suppose that $b \in C(a)$. Then $ab = ba$, so that
	\[a^3b = aaab = aaba = abaa = baaa = ba^3 \]
	showing that $b \in C(a^3)$, proving (1). To prove (2), suppose that 
	$b \in C(a^3)$. Then $a^3b = ba^3$; noting that because 
	$|a| = 5 \implies a^5 = e$, observe
	\[ ab = a^5ab = a^6b = a^3a^3b = a^3ba^3 = ba^3a^3 = ba^6 = baa^5 = ba\]
	showing that $b \in C(a)$, proving (2). Since $C(a) \subseteq C(a^3)$ and $C(a^3)
	\subseteq C(a)$, it follows that $C(a) = C(a^3)$.
\end{proof}


\subsection*{Problem 7}
Prove directly that the ring $\mathbb{Z}[x]$ is an integral domain. 
\begin{proof}
	It suffices to show that $\mathbb{Z}[x]$ is a commutative ring that has no 
	zero divisors. To this end, we will show that given $a, b \in \mathbb{Z}[x]$ with
	$ab = 0$, it follows that $a = 0$ or $b = 0$. Because $a$ is a polynomial with int
	eger coefficients, its leading coefficient in particular will be an integer, call
	it $m$; similarly, $b$ will have an integer as its leading coefficient, call it
	$n$. By multiplication of polynomials, it follows that the leading coefficient of
	$ab$ is equal to $mn$; since the leading coefficient of 0 is 0, and the 
	coefficients must match up, it follows that $mn = 0$. By the fact that 
	$\mathbb{Z}$ is an integral domain, we deduce that either $m$ or $n$ must be 
	zero, meaning that one of $a$ or $b$ must be equal to the zero polynomial. 
	Clearly, multiplication in $\mathbb{Z}[x]$ is commutative because multiplication 
	in $\mathbb{Z}$ is commutative. Thus $\mathbb{Z}[x]$ is an integral domain. 
\end{proof}

\subsection*{Problem 8}
Let $f(x)$ and $g(x)$ be polynomials with rational coefficients. Show
that if $f(n) = g(n)$ for all integers $n$, then $f(x) = g(x)$.
\begin{proof}
	Define $h(x) = f(x) - g(x)$. $h(x)$ is a polynomial over $\mathbb{Q}[x]$ because
	both $f(x)$ and $g(x)$ are polynomials over $\mathbb{Q}[x]$. Suppose that 
	$h(x) \neq 0$; then it follows that $h(x)$ has degree $k$ for some positive
	integer $k$. This means that $h(x)$ has at most $k$ zeros. Because $f(n) = g(n)$ 
	for all integers $n$, it follows that $h(n) = 0$ for all integers $n$, meaning 
	that $h$ has infinitely many zeros; but	this contradicts the fact that $h(x)$ 
	has degree $k$. Thus $h(x) = 0$, meaning that $f(x) = g(x)$.
\end{proof}

\subsection*{Problem 9}
Prove that the set of polynomials whose coefficients are all even is a
prime ideal in $\mathbb{Z}[x]$.
\begin{proof}
	Denote the set $A$. First, we will show that $A$ is an ideal. To this end, take
	$a, b \in A$. Because subtraction of even integers preserves parity, it follows
	that $a - b \in A$. Now take any polyomial $p \in \mathbb{Z}[x]$. Because
	multiplication of polynomials is commutative, it suffices to show that 
	$pa = ap \in A$. Observe that the each coefficient of $ap$ arises by
	multiplying one coefficient from $a$ and $p$. Every coefficient of $a$ is even,
	and the parity of the coefficients of $p$ does not matter because an even number
	multipled by an odd number is always even. Thus all of the coefficients of $ap$ 
	are even, meaning that $ap \in A$, so $A$ is an ideal. To show that it is prime,
	we will use the fact that $\mathbb{Z}[x]/A$ is an integral domain. It is 
	well-defined as a ring because $A$ is an ideal and it is commutative because 
	$\mathbb{Z}[x]$ is commutative, so we must show that the factor ring has no zero 
	divisors. To this end, observe that if $c + A$ and $d + A \in \mathbb{Z}[x]/A$ 
	are non-trivial elements, it follows that $c$ and $d$ each have at least one odd 
	coefficient because multiplication of an odd integer with another odd integer
	yields an odd integer, meaning that the product $(c + A)(d + A)$ will be 
	non-trivial (because $cd$ will have at least one part with an odd coefficient, 
	meaning that part will not be absorbed into $A$).

\end{proof}

\subsection*{Problem 10}
\begin{enumerate}
	\item Show that $g(x) = x^3 + x + 1$ is irreducible in $\mathbb{Z}_2[x]$.
	\item Show that the ideal $I = \langle g(x) \rangle$ is maximal in 
		$\mathbb{Z}_2[x]$ directly.
	\item Show that the factor ring $F = \frac{\mathbb{Z}_2[x]}{I}$ is a field. So 
		the subset $F^\times = F - \{0\}$ is a group under multiplication.
	\item What is the order of $x + 1$ in the group $F^\times$?
\end{enumerate}

\begin{proof}
	\qquad
	\begin{enumerate}
		\item 
			Observe that $g(0) = 1$ and $g(1) = 1$, meaning that $g(x)$ has no roots
			over $\mathbb{Z}_2[x]$; thus $g(x)$ is irreducible over $\mathbb{Z}_2[x]$
		\item
			Let $J$ be any ideal such that $I \subseteq J \subseteq \mathbb{Z}_2[x]$.
			We have that $\mathbb{Z}_2[x]$ is a principal ideal domain, meaning that 
			$J = \langle f(x) \rangle$ for some polynomial $f(x)$. Thus we have that
			$g(x) = f(x)h(x)$ for some polynomial $h(x)$. But because $g(x)$ is 
			irreducible, it follows that either $f(x)$ or $h(x)$ is a constant;
			in the first case, we have that $J = \mathbb{Z}_2[x]$, and in the second
			case we have that $J = I$. Thus $I$ is a maximal ideal by definition.
		\item 
			Because $I$ is a maximal ideal of $\mathbb{Z}_2[x]$, it follows that 
			the factor ring is a field; this means that the subset of nonzero elements
			of $F$ form a group under multiplication, as there are multiplicative
			inverses by virtue of the subset being a field.
		\item 
			The order of $x + 1$ is 7 in this group; we use the fact that
			$x^3 = -x^3 = x + 1$ to show this. By repeatedly multiplying $x+1$ by
			itself until we get 1, we notice that
			\[ (x+1)^2=x^2+2x+1=x^2+1 \]
			\[ (x+1)^3=x^3+x^2+x+1=x^2 \]
			\[ (x+1)^4=x^3+x^2=x^2+x+1 \]
			\[ (x+1)^5=x^3+1=x \]
			\[ (x+1)^6=x^2+x \]
			\[ (x+1)^7=x^3+x=1 \]
			Also, we can use the observation that $F - \{0\}$ has 7 elements, so the
			order of $x + 1$ must be 1 or 7 by Lagrange's theorem; clearly it is not
			the identity, so the order is 7.

	\end{enumerate}
\end{proof}

\subsection*{Extra Credit}
Let $R$ be a ring with $x^2 = x$ for all $x \in R$. Prove that $R$ is commutative.

\begin{proof}
	First, we make the observation that $R$ has characteristic 2. Note that for any 
	$r \in R$, we have that $(r + r) = (r + r)^2 = (2r)^2 = 4r^2 = 4r$, which means 
	that $r + r = 0$. Now take $x, y \in R$ and observe that $(x + y) = (x+y)^2 = 
	x^2 + y^2 + xy + yx = x + y + xy + yx$, meaning that $xy + yx = 0$ so that 
	$xy = -yx$; thus it follows that $xy = yx$ from the fact that the ring has 
	characteristic 2. We conclude that $R$ is commutative. 
\end{proof}


\end{document}

