\documentclass{article}
\usepackage[utf8]{inputenc}
\usepackage[english]{babel}
\usepackage[]{amsthm} 
\usepackage{enumitem}
\usepackage{array}
\usepackage{amsmath}
\usepackage[]{amssymb} 
\usepackage{gensymb}
\DeclareMathOperator{\lcm}{lcm}

\title{HW 5 - MATH403}
\author{Danesh Sivakumar}
\date\today


\begin{document}
\maketitle 

\subsection*{Problem 1 (Chapter 7, Exercise 6)}
Suppose that $a$ has order 15. Find all of the left cosets of $\langle a^5 \rangle$ in $\langle a \rangle$.

\begin{proof}
We know that because $a$ has order 15, $\langle a \rangle$ also has order 15. We also know that the order of $\langle a^5 \rangle$ is $15/\gcd{(5, 15)} = 15/5 = 3$ by the cyclic order formula. This means that there are $15/3 = 5$ distinct cosets of $\langle a^5 \rangle$ in $\langle a \rangle$; we claim that these are $\langle a^5 \rangle$, $a\langle a^5 \rangle$, $a^2\langle a^5 \rangle$, $a^3\langle a^5 \rangle$ and $a^4\langle a^5 \rangle$. To this end, notice: 
\[\langle a^5 \rangle = \{e, a^5, a^{10}\}\]
\[a\langle a^5 \rangle = \{a, a^6, a^{11}\}\]
\[a^2\langle a^5 \rangle = \{a^2, a^7, a^{12}\}\]
\[a^3\langle a^5 \rangle = \{a^3, a^8, a^{13}\}\]
\[a^4\langle a^5 \rangle = \{a^4, a^9, a^{14}\}\]
and all of these cosets form a disjoint union for $\langle a \rangle$, so they are indeed the only left cosets of $\langle a^5 \rangle$ in $\langle a \rangle$.
\end{proof}


\subsection*{Problem 2 (Chapter 7, Exercise 8)}
Give an example of a group $G$ and subgroups $H$ and $K$ such that $HK = \{h \in H,\, k \in K\}$ is not a subgroup of $G$.

\begin{proof}
Take $G = S_3$, $H = \langle (12) \rangle$ and $K = \langle (23) \rangle$. Then $H$ and $K$ are subgroups of $G$, but $HK = \{1, (12), (23), (132)\}$; because this set is of size 4, it doesn't divide $|G| = 6$, so it cannot be a subgroup of $G$ by Lagrange's theorem. 

\end{proof}

\subsection*{Problem 3 (Chapter 7, Exercise 12)}
Let $a$ and $b$ be nonidentity elements of different orders in a group $G$ of order 155. Prove that the only subgroup of $G$ that contains $a$ and $b$ is $G$ itself.
\begin{proof}
Suppose that $H$ is a subgroup of $G$ containing both $a$ and $b$. By Lagrange's theorem, the only possible orders of $H$ are 5, 31, or 155 (the trivial subgroup is not allowed because $a$ and $b$ are nonidentity elements). We will first prove that any group of prime order is cyclic, and that all of its nonidentity elements have the same order; the result will follow then. To this end, let $G$ be a group of order $p$, where $p$ is prime. Consider the subgroup generated by any arbitrary nonidentity $g \in G$; by Lagrange's theorem, the order of this subgroup must divide $p$; but since $p$ is prime and $g \neq e$, this means $|\langle g \rangle| = p$, so that $\langle g \rangle = G$. Also, take any arbitrary nonidentity $a \in \langle g \rangle$; by similar reasoning, $|\langle a \rangle| = p$, meaning that any nonidentity element is a generator. Because 5 and 31 are primes, it follows that every nonidentity element in the subgroup of order 5 has order 5 and every nonidentity element in the subgroup of order 31 has order 31. This means that the only possibility for the order of $H$ is 155, which implies that $H = G$.
\end{proof}

\subsection*{Problem 4 (Chapter 7, Exercise 26)}
Suppose that $G$ is a group with more than one element and $G$ has no proper, nontrivial subgroups. Prove that $|G|$ is prime. 
\begin{proof}
Suppose $|G| \geq 2$, with $|G|$ possibly $\infty$. If $G$ has no proper, nontrivial subgroups, then the only subgroups of $G$ are $e$ and $G$ itself. Take arbitrary nonidentity $a \in G$ and consider $\langle a \rangle$; clearly $\langle a \rangle \neq e$, so $\langle a \rangle = G$, meaning that $G$ is cyclic. Now assume $|G| = \infty$; this means that $G \cong \mathbb{Z}$, which is a contradiction because $\mathbb{Z}$ has nontrivial subgroups. Thus $|G| = n$ for some finite $n$. Since $G$ is cyclic, by the fundamental theorem of cyclic groups it follows that there is exactly one subgroup of order $d$ for each positive divisor $d$ of $n$, and these are the only subgroups of $G$. Thus, if $d \neq 1$ or $d \neq n$, then $G$ has proper nontrivial subgroups, which is a contradiction; thus the only divisors of $n$ are $n$ and $1$, meaning that $n$ is prime.
\end{proof}

\subsection*{Problem 5 (Chapter 7, Exercise 28)}
Let $G$ be a group of order 25. Prove that $G$ is cyclic or $g^5 = e$ for all $g$ in $G$. Generalize to any group of order $p^2$ where $p$ is prime.
\begin{proof}
If $G$ is cyclic, we are done. Suppose that $G$ is not cyclic; we claim that for all $g \in G, g^5 = e$. If $g = e$, then clearly $g^5 = e$, so suppose $g \neq e$. By Lagrange's theorem $|g|$ divides 25, so that $g$ could possibly be 1, 5, or 25. However, $|g| \neq 1$ because $g \neq e$ and $|g| \neq 25$ because $G$ is not cyclic; thus $|g| = 5$ so that $g^5 = e$. An analogous result holds true for any prime $p$; if $G$ is a group of order $p^2$, then $p$ is cyclic or $g^p = e$ for all $g \in G$ through the same line of reasoning because the only divisors of $p^2$ are $1$, $p$ and $p^2$ by the fundamental theorem of arithmetic.
\end{proof}


\subsection*{Problem 6 (Chapter 7, Exercise 32)}
Determine all finite subgroups of $\mathbb{C}^*$, the group of nonzero complex numbers under multiplication.
\begin{proof}
Suppose $H \leq \mathbb{C}^*$, with $|H|$ finite, say $|H| = n$. By Lagrange's theorem, it follows that for any $z \in H$, $z^{|H|} = z^n = 1$, because 1 is the identity in $\mathbb{C}^*$. However, the solutions of $z^n = 1$ are by definition the $n$th roots of unity, which implies that $H$ consists precisely of $n$th roots of unity. These roots certainly form a subgroup, because (1) they are closed under the operation of multiplication and (2) inverses exists. To show (1), let $z_1 = e^{a2\pi/n}$ and $z_2 = e^{b2\pi/n}$, with $a, b < n$. Notice that $z_1z_2 = e^{(a+b)2\pi/n} = e^{(a+b \mod{n})2\pi/n}$, and $0 < a+b \mod{n} < n$ so that $z_1z_2 \in H$. To show (2), notice that $z_1^{-1} = e^{(n-a)2\pi/n}$, and $z_1z_1^{-1} = e^{a2\pi/n}e^{(n-a)2\pi/n} = e^{n2\pi/n} = e^{2\pi} = 1$ (multiplication is commutative).
\end{proof}

\subsection*{Problem 7 (Chapter 7, Exercise 40)}
Prove that a group of order 63 must have an element of order 3.
\begin{proof}
By Lagrange's theorem, the only possible orders for elements are 1, 3, 7, 9, 21, and 63 because those are the divisors of 63. Pick a non-identity element $g$. If $|g| = 3$, we are done. If $|g| = 9$, then $|g^3| = 3$. If $|g| = 21$, then $|g^7| = 3$. If $|g| = 63$, then $|x^21| = 3$. If none of these are the case, then there are 62 non-identity elements of order 7, which is impossible because 62 is not a multiple of $\phi(7) = 6$.
\end{proof}

\subsection*{Problem 8 (Chapter 7, Exercise 46)}
Prove that a group of order 12 must have an element of order 2.
\begin{proof}
By Lagrange's theorem, the only possible orders for elements are 1, 2, 3, 4, 6 and 12 because those are the divisors of 12. Pick a non-identity element $g$. If $|g| = 2$, we are done. If $|g| = 4$, then $|g^2| = 2$. If $|g| = 6$, then $|g^3| = 2$. If $|g| = 12$, then $|g^6| = 2$. If none of these are the case, then there are 11 non-identity elements of order 3, which is impossible because 11 is not a multiple of $\phi(3) = 2$.
\end{proof}

\subsection*{Problem 9 (Chapter 7, Exercise 62)}
Calculate the orders of the following
\begin{enumerate}[label=\alph*.]
\item The group of rotations of a regular tetrahedron 
\item The group of rotations of a regular octahedron
\item The group of rotations of a regular dodecahedron
\item The group of rotations of a regular icosahedron
\end{enumerate}
\begin{proof}
We proceed via the Orbit-Stabilizer Theorem.
\begin{enumerate}[label=\alph*.]
\item \textbar stabilizer\textbar = 3 because rotating about any vertex a multiple of $2\pi/3$ radians preserves structure and \textbar orbit\textbar = 4 because there are four vertices, so order of rotations is $3 \cdot 4 = 12$. 
\item \textbar stabilizer\textbar = 4 because rotating about any vertex a multiple of $\pi/2$ radians preserves structure and \textbar orbit\textbar = 6 because there are six vertices, so order of rotations is $4 \cdot 6 = 24$.
\item \textbar stabilizer\textbar = 5 because rotation about any face a multiple of $2\pi/5$ radians preserves structure and \textbar orbit\textbar = 12 because there are twelve faces, so order of rotations is $5 \cdot 12 = 60$.
\item \textbar stabilizer\textbar = 5 because rotation about any vertex a multiple of $2\pi/5$ radians preserves structure, and \textbar orbit\textbar = 12 because there are twelve vertices, so order of rotations is $5 \cdot 12 = 60$.
\end{enumerate}
\end{proof}

\subsection*{Problem 10 (Chapter 8, Exercise 8)}
Is $\mathbb{Z}_3 \oplus \mathbb{Z}_9$ isomorphic to $\mathbb{Z}_{27}$?
\begin{proof}
Notice that 3 and 9 are not relatively prime, so that $\mathbb{Z}_3 \oplus \mathbb{Z}_9$ cannot be isomorphic to $\mathbb{Z}_{3 \cdot 9} = \mathbb{Z}_{27}$. Also, if there did exist an isomorphism, there would have to be an element of order 27 in $\mathbb{Z}_3 \oplus \mathbb{Z}_9$ because isomorphism preserves order; but there are no elements $z_1 \in \mathbb{Z}_3$ and $z_2 \in \mathbb{Z}_9$ such that the least common multiple of their orders is 27; thus an isomorphism cannot exist.
\end{proof}


\end{document}
