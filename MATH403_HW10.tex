\documentclass{article}
\usepackage[utf8]{inputenc}
\usepackage[english]{babel}
\usepackage[]{amsthm} 
\usepackage{enumitem}
\usepackage{array}
\usepackage{amsmath}
\usepackage{amssymb} 
\usepackage{gensymb}
\DeclareMathOperator{\lcm}{lcm}

\title{HW 10 - MATH403}
\author{Danesh Sivakumar}
\date{April 24th, 2022}


\begin{document}
\maketitle 

\subsection*{Problem 1 (Chapter 13, Exercise 52)}
Give an example of an infinite integral domain that has characteristic 3.
\begin{proof}
$\mathbb{Z}_3[x]$ is an example; notice that $\mathbb{Z}_3$ is an integral domain, so that $\mathbb{Z}_3[x]$ is also an integral domain. It follows from Theorem 13.3 that this integral domain has characteristic 3.
\end{proof}


\subsection*{Problem 2 (Chapter 13, Exercise 56)}
Find all solutions of $x^2 - x + 2 = 0$ over $\mathbb{Z}_3[i]$
\begin{proof}
Note that the elements of $\mathbb{Z}_3[i]$ are $\{0, 1, 2, i, 1 + i, 2 + i, 2i, 1 + 2i, 2 + 2i\}$. Testing each of these with our polynomial yields: \\
$(0)^2 - (0) + 2 = 2 \neq 0$ \\  $(1)^2 - (1) + 2 = 2 \neq 0$ \\ $(2)^2 - (2) + 2 = 1 \neq 0$ \\ $(i)^2 - (i) + 2 = 1 + 2i \neq 0$ \\ $(1 + i)^2 - (1 + i) + 2 = 1 + i \neq 0$ \\ $(2 + i)^2 - (2 + i) + 2 = 0$ \\ $(2i)^2 - (2i) + 2 = 1 + i \neq 0$ \\ $(1 + 2i)^2 - (1 + 2i) + 2 = 1 + 2i \neq 0$ \\ $(2 + 2i)^2 - (2 + 2i) + 2 = 0$ \\
Thus the solutions are $x = 2 + i$ and $x = 2 + 2i$
\end{proof}

\subsection*{Problem 3 (Chapter 13, Exercise 64)}
Suppose that $a$ and $b$ belong to a field of order $8$ and that $a^2 + ab + b^2 = 0$. Prove that $a = 0$ and $b = 0$. Do the same when the field has order $2^n$ with $n$ odd.
\begin{proof}
Suppose that $a = b$; then $a^2 + ab + b^2 = 3a^2 = a^2 = 0$. This implies that $a = 0$ and $b = 0$. Now, toward a contradiction suppose that $a \neq b$ and without loss of generality $a \neq 0$. Observe that $0 = (a-b)(a^2+ab+b^2) = a^3 - b^3$ by the difference of powers formula. Thus $a^3 = b^3$. Now observe that the order of the field $F \setminus \{0\}$ is $2^n - 1$; the fact that $a^3 = b^3$ implies that $a^{-3}b^3 = 1$, meaning that $(a^{-1}b)^3 = 1$. But $3$ never divides $2^n - 1$ for $n$ odd. To prove this, we will show that $3$ always divides $2^n + 1$ for $n$ odd; then because $2^n - 1$ and $2^n + 1$ have a difference of 2, both cannot be divisible by $3$. We prove this by induction; observe that $2^1 + 1 = 3$ is divisible by $3$. Now suppose $2^{2n+1} + 1 = 3k$; it follows that $4(3k) - 3 = 3(4k - 1) = 3j = 2^{2n+1} + 1$. Thus, we have that the order of $a^{-1}b$ is 1, and so $a^{-1}b = 1$, meaning $a = b$, which is a contradiction; thus $a = b$ and $a = 0$ and $b = 0$.
\end{proof}

\subsection*{Problem 4 (Chapter 14, Exercise 16)}
If $A$ and $B$ are ideals of a commutative ring $R$ with unity and $A + B = R$, show that $A \cap B = AB$
\begin{proof} 
Take $x \in A \cap B$. Because $R = A + B$, it follows that $1 = a + b$ for $a \in A$ and $b \in B$. This means that $x = 1 \cdot x = (a + b) \cdot x = ax + bx = ax + xb \in AB$ because $a \in A$, $x \in A$, $x \in B$ and $b \in B$. Now take $x \in AB$. It follows that $x = \sum_{i=1}^{n} a_ib_i$ for $a_i \in A$ and $b_i \in B$ and some $n$. Because $A$ is an ideal, $a_ib_i \in A$. Because $B$ is an ideal, $a_ib_i \in B$. Because $a_ib_i \in A$ and $a_ib_i \in B$ for all i, it follows that $x \in A \cap B$.
\end{proof}

\subsection*{Problem 5 (Chapter 14, Exercise 32)}
Show that $A = \{(3x, y) \mid x, y \in \mathbb{Z}\}$ is a maximal ideal of $\mathbb{Z} \oplus \mathbb{Z}$. Generalize. What happens if $3x$ is replaced by $4x$? Generalize.
\begin{proof} 
Suppose $B$ is an ideal such that $A \subset B$. We claim that $B = \mathbb{Z} \oplus \mathbb{Z}$. To this end, suppose there exists an element $(m, n) \in B$ such that $(m, n) \notin A$. It follows that $m$ is not a multiple of 3, so that $\gcd{(m, 3)} = 1$ because $3$ is prime. Then by Bezout's lemma, we can find integers $a, b$ such that $3a + mb = 1$, meaning that $(1, 1) \in B$ so that $B = \mathbb{Z} \oplus \mathbb{Z}$. This reasoning extends to any prime $p$. However, this does not work with $4x$ because $A \subset \{(2x, y) \mid x, y \in \mathbb{Z}\} \subset \mathbb{Z} \oplus \mathbb{Z}$. In general, this does not work with composite numbers.
\end{proof}


\subsection*{Problem 6 (Chapter 14, Exercise 34)}
Let $R = \mathbb{Z}_8 \oplus \mathbb{Z}_{30}$. Find all maximal ideals of $R$, and for each maximal ideal $I$, identify the size of the field $R/I$.
\begin{proof}
We proceed by taking the direct product of the ideal of one component and the other component. \\
$I_1 = 2\mathbb{Z}_8 \oplus \mathbb{Z}_{30}$ \\ $I_2 = 2\mathbb{Z}_8 \oplus 2\mathbb{Z}_{30}$ \\ $I_3 = \mathbb{Z}_8 \oplus 3\mathbb{Z}_{30}$ \\ $I_4 = \mathbb{Z}_8 \oplus 5\mathbb{Z}_{30}$ \\
By calculating the orders of the fields, we deduce that $|I_1| = 2$, $|I_2| = 2$, $|I_3| = 3$ and $|I_4| = 5$
\end{proof}

\subsection*{Problem 7 (Chapter 14, Exercise 54)}
List the elements of the field $\mathbb{Z}_2[x]/\langle x^2 + x + 1 \rangle$, and make an addition and multiplication table for the field.
\begin{proof}
Note that the only possibilities are $0, 1, x, x + 1, x^2, x^2 + 1, x^2 + x$ because $\langle x^2 + x + 1 \rangle$ has degree 2. But the fact that $x^2 + x + 1 = 0$ implies that some of these elements go away; namely $x^2 + x = x^2 + x - (x^2 + x + 1) = -1 = 1$, $x^2 + 1 = x^2 + 1 - (x^2 + x + 1) = -x = x$, and $x^2 = x^2 - (x^2 + x + 1) = -(x+1) = x+1$. This means that the only elements of the field are $0, 1, x, x+1$. The tables are listed below:
\[
    \setlength{\extrarowheight}{3pt}
    \begin{array}{l|*{4}{l}}
     +  & 0   & 1   & x  & x+1  \\
    \hline
    0   & 0   & 1   & x  & x+1  \\
    1   & 1   & 0   & x+1 & x  \\
    x   & x   & x+1  & 0  & 1  \\
    x+1  & x+1  & x   & 1  & 0  \\
    \end{array} 
\]

\[
    \setlength{\extrarowheight}{3pt}
    \begin{array}{l|*{4}{l}}
     \times  & 0   & 1   & x  & x+1  \\
    \hline
    0   & 0   & 0   & 0  & 0  \\
    1   & 0   & 1   & x & x+1  \\
    x   & 0   & x  & x+1  & 1  \\
    x+1  & 0  & x+1   & 1  & x  \\
    \end{array} 
\]

\end{proof}

\subsection*{Problem 8 (Chapter 14, Exercise 58)}
Show that $\mathbb{Z}[i]/\langle 1 - i \rangle$ is a field. How many elements does this field have? 
\begin{proof}
Note that because $1 - i = 0$, it follows that $1 = i$, so that $1 = -1$ and thus $2 = 0$. We then have that for any $x + yi \in \mathbb{Z}[i]$, that $x + yi + \langle 1 - i \rangle = ki + \langle 1 - i \rangle$ for $k = 1$ or $k = 0$. This means that $\mathbb{Z}[i]/\langle 1 - i \rangle$ is a field with two distinct elements, namely $\langle 1 - i \rangle$ and $i + \langle 1 - i \rangle$
\end{proof}

\subsection*{Problem 9 (Chapter 14, Exercise 66)}
Let $R = \mathbb{Z}[\sqrt{-5}]$ and let $I = \{ a + b\sqrt{-5} \mid a, b \in \mathbb{Z}, a - b \text{ is even}\}$. Show that $I$ is a maximal ideal of $R$.
\begin{proof} 
Suppose $J$ is an ideal such that $I \subset J \subset R$. We have that $J$ contains an element $a + b\sqrt{-5}$ such that the parity of $a$ and $b$ is different. Case 1: $a$ odd and $b$ even; $(2m + 1) + 2n\sqrt{-1}$. Case 2: $a$ even and $b$ odd; $2m + (2n + 1)\sqrt{-1}$. Note that adding $1$ to both elements yields an element in $I$. Thus, it follows that $1 = [(2m + 1) + 2n\sqrt{-5}] - [2m + 2n\sqrt{-5}] = [(2m + 1) + (2n+1)\sqrt{-5}] - [2m + (2n+1)\sqrt{-5}] \in J$, meaning that $J = R$; this means that $I$ is a maximal ideal of $R$.  
\end{proof}

\subsection*{Problem 10 (Chapter 14, Exercise 70)}
Let $R = \{ (a_1, a_2, a_3, \cdots\}$, where each $a_i \in \mathbb{Z}$. Let $I = \{ (a_1, a_2, a_3, \cdots\}$, where only a finite number of terms are nonzero. Prove that $I$ is not a principal ideal of $R$.
\begin{proof}
Suppose for the sake of contradiction that there is exists a sequence $a = \{ (a_1, a_2, a_3, \cdots\}$ such that $I = \langle a \rangle$. Because there are only a finite number of nonzero terms, there exists an index $m$ such that $a_n = 0$ for all $n \geq m$. Now consider the sequence $b = \{ (a_1, a_2, a_3, \cdots\, a_m, 1, 0, 0, \cdots\}$. It follows that this sequence is an element of $I$, but there cannot exist an $r \in R$ such that $b = ar$ because this implies $1 = a_{m+1}r_{m+1}$, which cannot happen because $a_{m+1} = 0$. Thus $I$ cannot be a principal ideal of $R$.
\end{proof}

\subsection*{Problem 11 (Chapter 15, Exercise 56)}
Let $\mathbb{Q}[\sqrt{2}] = \{ a + b\sqrt{2} \mid a, b \in \mathbb{Q}\}$ and $\mathbb{Q}[\sqrt{5}] = \{ a + b\sqrt{5} \mid a, b \in \mathbb{Q}\}$. Show that these two rings are not ring-isomorphic.
\begin{proof}
Suppose such an isomorphism $\phi \colon \mathbb{Q}[\sqrt{2}] \to \mathbb{Q}[\sqrt{5}]$ exists. Then $\phi(\sqrt{2}) = a + b\sqrt{5} \implies \phi(2) = a^2 + 2ab\sqrt{5} + 5b^2$. However, $phi(2) = 2\phi(1) = 2$. This is a contradiction, because $2$ is rational but $a^2 + 2ab\sqrt{5} + 5b^2$ cannot be rational because of the additional $\sqrt{5}$ term. Thus, no such isomorphism exists.
\end{proof}

\subsection*{Problem 12 (Chapter 15, Exercise 66)}
Let $R = \bigg\{ \begin{bmatrix} a & b \\ b & a \end{bmatrix} \bigg| \text{ }a, b \in \mathbb{Z}\bigg\}$, and let $\phi$ be the mapping that takes $\begin{bmatrix} a & b \\ b & a \end{bmatrix}$ to $a - b$.
\begin{enumerate}[label = (\alph*)]
    \item Show that $\phi$ is a homomorphism.
    \item Determine the kernel of $\phi$.
    \item Show that $R/\text{Ker} \phi$ is isomorphic to $\mathbb{Z}$
    \item Is $\text{Ker} \phi$ a prime ideal?
    \item Is $\text{Ker} \phi$ a maximal ideal?
    
\end{enumerate}
\begin{proof}
\begin{enumerate}[label = (\alph*)]
    \item \[ \phi\left( \begin{bmatrix} a & b \\ b & a \end{bmatrix} + \begin{bmatrix} c & d \\ d & c \end{bmatrix}\right) = \phi\left( \begin{bmatrix} a+c & b+d \\ b+d & a+c \end{bmatrix}\right) = a + c - b - d = a - b + c - d\] 
    \[= \phi\left( \begin{bmatrix} a & b \\ b & a \end{bmatrix}\right) + \phi\left( \begin{bmatrix} c & d \\ d & c \end{bmatrix}\right)\]
    \[ \phi\left( \begin{bmatrix} a & b \\ b & a \end{bmatrix} \begin{bmatrix} c & d \\ d & c \end{bmatrix}\right) = \phi\left( \begin{bmatrix} ac + bd & ad + bc \\ ad + bc & ac + bd \end{bmatrix}\right) = (ac + bd) - (ad + bc) = (a-b)(c-d) = \] 
    \[= \phi\left( \begin{bmatrix} a & b \\ b & a \end{bmatrix}\right)\phi\left( \begin{bmatrix} c & d \\ d & c \end{bmatrix}\right)\]
    so that $\phi$ is operation preserving in multiplication and addition.
    \item Observe that the kernel is the set of matrices such that $a - b = 0$, which occurs only when $a = b$; thus Ker$\phi = \bigg\{ \begin{bmatrix} a & a \\ a & a \end{bmatrix} \bigg| \text{ } a \in \mathbb{Z}\bigg\}$
    \item Because $\phi$ is an onto homomorphism (it can take any value in $\mathbb{Z}$), it follows by the First Isomorphism Theorem that $R/\text{Ker} \phi \cong \mathbb{Z}$.
    \item Yes, because $\mathbb{Z}$ is an integral domain.
    \item No, because $\mathbb{Z}$ is not a field.
\end{enumerate}


\end{proof}



\end{document}
