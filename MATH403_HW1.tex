\documentclass{article}
\usepackage[utf8]{inputenc}
\usepackage[english]{babel}
\usepackage[]{amsthm} 
\usepackage{enumitem}
\usepackage{array}
\usepackage{amsmath}
\usepackage[]{amssymb} 

\title{HW 1 - MATH403}
\author{Danesh Sivakumar}
\date\today


\begin{document}
\maketitle 


\subsection*{Problem 1 (Chapter 2, Exercise 4)}
Which of the following sets are closed under the given operation?
\begin{enumerate}[label=(\alph*)]
\item \{0, 4, 8, 12\} addition mod 16
\item \{0, 4, 8, 12\} addition mod 15
\item \{1, 4, 7, 13\} multiplication mod 15
\item \{1, 4, 5, 7\} multiplication mod 9
\end{enumerate}
\begin{proof}
$ $
\begin{enumerate}[label=(\alph*)]
\item
Given the Cayley table:
\[
    \setlength{\extrarowheight}{3pt}
    \begin{array}{l|*{4}{l}}
        & 0   & 4   & 8  & 12  \\
    \hline
    0   & 0   & 4   & 8  & 12  \\
    4   & 4   & 8   & 12 & 0  \\
    8   & 8   & 12  & 0  & 4  \\
    12  & 12  & 0   & 4  & 8  \\
    \end{array} 
\]

We observe that all entries in the table are in the set; thus the group is indeed closed.

\item 
Note that $(4+12)$ mod $15 = 1 \notin G$; thus the group is not closed.

\item 
Given the Cayley table:
\[
    \setlength{\extrarowheight}{3pt}
    \begin{array}{l|*{4}{l}}
        & 1   & 4   & 7  & 13  \\
    \hline
    1   & 1   & 4   & 7  & 13  \\
    4   & 4   & 1   & 13 & 7  \\
    7   & 7   & 13  & 4  & 1  \\
    13  & 13  & 7   & 1  & 4  \\
    \end{array} 
\]

We observe that all entries in the table are in the set; thus the group is indeed closed.

\item
Note that $(4 \cdot 5)$ mod $9 = 2 \notin G$; thus the group is not closed.

\end{enumerate}

\end{proof}


\subsection*{Problem 2 (Chapter 2, Exercise 16)}
Show that the set \{5, 15, 25, 35\} is a group under multiplication modulo 40. What is the identity element of this group? Can you see any relationship between this group and $U(8)$?
\begin{proof}
Given the Cayley table of this group:
\[
    \setlength{\extrarowheight}{3pt}
    \begin{array}{l|*{4}{l}}
        & 5   & 15   & 25  & 35  \\
    \hline
    5   & 25   & 35   & 5  & 15  \\
    15   & 35   & 25   & 15 & 5  \\
    25   & 5   & 15  & 25  & 35  \\
    35  & 15  & 5   & 35  & 25  \\
    \end{array} 
\]

We observe that all entries in the table are in the set; thus the group is indeed closed. Furthermore, note that 25 is the identity; that is, it is the element $e$ with the property that for any $a \in G$, $a \cdot e = a$. Now, given the Cayley table of $U(8)$:
\[
    \setlength{\extrarowheight}{3pt}
    \begin{array}{l|*{4}{l}}
        & 1   & 3   & 5  & 7  \\
    \hline
    1   & 1   & 3   & 5  & 7  \\
    3   & 3   & 1   & 7 & 5  \\
    5   & 5   & 7  & 1  & 3  \\
    7  & 7  & 5   & 3  & 1  \\
    \end{array} 
\]
We observe that each element of the original group corresponds to an element of $U(8)$; namely, 5 corresponds to 5, 15 corresponds to 7, 25 corresponds to 1, and 35 corresponds to 3.

\end{proof}

\subsection*{Problem 3 (Chapter 2, Exercise 32)}
Construct a Cayley table for $U(12)$.
\begin{proof}
\[
    \setlength{\extrarowheight}{3pt}
    \begin{array}{l|*{4}{l}}
        & 1   & 5   & 7  & 11  \\
    \hline
    1   & 1   & 5   & 7  & 11  \\
    5   & 5   & 1   & 11 & 7  \\
    7   & 7   & 11  & 1  & 5  \\
    11  & 11  & 7   & 5  & 1  \\
    \end{array} 
\]
\end{proof}

\subsection*{Problem 4 (Chapter 2, Exercise 36)}
Let $a$ and $b$ belong to a group $G$. Find an $x$ in $G$ such that $xabx^{-1} = ba$.

\begin{proof}
Suppose $a, b, x \in G$ with the property that $xabx^{-1} = ba$. Then

\[ xabx^{-1} = ba \]
\[ xabx^{-1}x = bax \]
\[ xabe = bax \]
\[ xab = bax \]

Matching terms, we get that $x=b$ works. We know that $b^{-1} \in G$ because $b \in G$, so:

\[ babb^{-1} = bae = ba \]

Similarly, $x=a^{-1}$ works:

\[ a^{-1}aba = eba = ba \]


\end{proof}


\subsection*{Problem 5 (Chapter 2, Exercise 46)}
Prove that the set of all $3 \times 3$ matrices with real entries of the form

\[A = \begin{bmatrix}
1 & a & b \\
0 & 1 & c \\
0 & 0 & 1
\end{bmatrix} \]

is a group.

\begin{proof}

Multiplication is defined as follows:

\[
\begin{bmatrix}
    1 & a & b \\
    0 & 1 & c \\
    0 & 0 & 1
    \end{bmatrix}
    \cdot
    \begin{bmatrix}
    1 & a' & b' \\
    0 & 1 & c' \\
    0 & 0 & 1
    \end{bmatrix}
    =
    \begin{bmatrix}
    1 & a + a' & b' + ac' + b \\
    0 & 1 & c' + c \\
    0 & 0 & 1
    \end{bmatrix}
\]

It is clear that $a+a'$, $b'+ac'+b$, and $c'+c$ are each real valued; thus the set is closed under multiplication. 

We must first show that the set has an identity element; observe that the identity matrix $I_3$ is the identity in this group, because:

\[
\begin{bmatrix}
    1 & a & b \\
    0 & 1 & c \\
    0 & 0 & 1
    \end{bmatrix}
    \cdot
    \begin{bmatrix}
    1 & 0 & 0 \\
    0 & 1 & 0 \\
    0 & 0 & 1
    \end{bmatrix}
    =
    \begin{bmatrix}
    1 & a & b \\
    0 & 1 & c \\
    0 & 0 & 1
    \end{bmatrix}
\]

and

\[
\begin{bmatrix}
    1 & 0 & 0 \\
    0 & 1 & 0 \\
    0 & 0 & 1
    \end{bmatrix}
    \cdot
    \begin{bmatrix}
    1 & a & b \\
    0 & 1 & c \\
    0 & 0 & 1
    \end{bmatrix}
    =
    \begin{bmatrix}
    1 & a & b \\
    0 & 1 & c \\
    0 & 0 & 1
    \end{bmatrix}
\]

Thus $A \cdot e = e \cdot A = A$, with $e = I_3$

Now, we must show that inverses exist. Indeed, by equating coefficients in the definition of multiplication, we get:

\[
A^{-1} = 
\begin{bmatrix}
    1 & -a & -b+ac \\
    0 & 1 & -c \\
    0 & 0 & 1
    \end{bmatrix}
\]

which is in the set, as $-a$, $-b+ac$ and $-c$ are real valued. Multiplying this out yields:

\[
\begin{bmatrix}
    1 & a & b \\
    0 & 1 & c \\
    0 & 0 & 1
    \end{bmatrix}
    \cdot
    \begin{bmatrix}
    1 & -a & -b+ac \\
    0 & 1 & -c \\
    0 & 0 & 1
    \end{bmatrix}
    =
    \begin{bmatrix}
    1 & 0 & 0 \\
    0 & 1 & 0 \\
    0 & 0 & 1
    \end{bmatrix}
\]

and

\[
\begin{bmatrix}
    1 & -a & -b+ac \\
    0 & 1 & -c \\
    0 & 0 & 1
    \end{bmatrix}
    \cdot
    \begin{bmatrix}
    1 & a & b \\
    0 & 1 & c \\
    0 & 0 & 1
    \end{bmatrix}
    =
    \begin{bmatrix}
    1 & 0 & 0 \\
    0 & 1 & 0 \\
    0 & 0 & 1
    \end{bmatrix}
\]

Thus $A \cdot A^{-1} = A^{-1} \cdot A = e$

Lastly, we must demonstrate the associative property. Indeed, observe that:

\[
\left(\begin{bmatrix}
    1 & a & b \\
    0 & 1 & c \\
    0 & 0 & 1
    \end{bmatrix}
    \cdot
    \begin{bmatrix}
    1 & d & e \\
    0 & 1 & f \\
    0 & 0 & 1
    \end{bmatrix}\right)
    \cdot
    \begin{bmatrix}
    1 & g & h \\
    0 & 1 & i \\
    0 & 0 & 1
    \end{bmatrix}
\]

\[
=
\begin{bmatrix}
    1 & a+d & e+af+b \\
    0 & 1 & c+f \\
    0 & 0 & 1
    \end{bmatrix}
    \cdot
    \begin{bmatrix}
    1 & g & h \\
    0 & 1 & i \\
    0 & 0 & 1
    \end{bmatrix}
\]

\[
=
\begin{bmatrix}
    1 & g+a+d & h+i(a+d)+e+af+b \\
    0 & 1 & f+c+i \\
    0 & 0 & 1
    \end{bmatrix}
\]

and

\[
\begin{bmatrix}
    1 & a & b \\
    0 & 1 & c \\
    0 & 0 & 1
    \end{bmatrix}
    \cdot
    \left(\begin{bmatrix}
    1 & d & e \\
    0 & 1 & f \\
    0 & 0 & 1
    \end{bmatrix}
    \cdot
    \begin{bmatrix}
    1 & g & h \\
    0 & 1 & i \\
    0 & 0 & 1
    \end{bmatrix}\right)
\]

\[
=
\begin{bmatrix}
    1 & a & b \\
    0 & 1 & c \\
    0 & 0 & 1
    \end{bmatrix}
    \cdot
    \begin{bmatrix}
    1 & d+g & h+id+e \\
    0 & 1 & f+i \\
    0 & 0 & 1
    \end{bmatrix}
\]

\[
=
\begin{bmatrix}
    1 & g+a+d & h+e+di+a(f+i)+b \\
    0 & 1 & f+c+i \\
    0 & 0 & 1
    \end{bmatrix}
\]

Thus, given matrices $A$, $B$ and $C$, it follows that $(AB)C = A(BC) = ABC$.

All three of the group axioms are satisfied, so this set under multiplication forms a group.


\end{proof}


\subsection*{Problem 6 (Chapter 2, Exercise 48)}
In a finite group, show that the number of nonidentity elements that satisfy the equation $x^5 = e$ is a multiple of 4. If the stipulation that the group be finite is omitted, what can you say about the number of nonidentity elements that satisfy the equation $x^5 = e$?

\begin{proof}

Suppose that for $a \in G$, we have $a^5 = e$ with $a \neq e$. \\
Then, it follows that $(a^2)^5 = (a^5)^2 = e^2 = e$. Suppose FSOC $a^2 = e$, then $(a^2)^2 = a^4 = e^2 = e = a^5$, implying that $a = e$, which is a contradiction, so, $a^2 \neq e$. \\
Similarly, it follows that $(a^3)^5 = (a^5)^3 = e^3 = e$. Suppose FSOC $a^3 = e$, then $(a^3)^2 = a^6 = e^2 = e = a^5$, implying that $a = e$, which is a contradiction, so, $a^3 \neq e$. \\
Similarly, it follows that $(a^4)^5 = (a^5)^4 = e^4 = e$. Suppose FSOC $a^4 = e$, then $e = a^4 = a^5$, implying that $a = e$, which is a contradiction, so, $a^4 \neq e$. \\
We claim that for distinct $i, j \in \{1, 2, 3, 4\}, a^i \neq a^j$. To this end, suppose FSOC that $a^i = a^j$. This is equivalent to $a^{i-j} = e$. WLOG assume $i > j$, then $i-j \in \{1, 2, 3\}$. We previously showed that $a, a^2, a^3 \neq e$, so $a^{i-j} \neq e$, which is a contradiction; thus, $a^i \neq a^j$. \\
Thus, we deduce that $\{a, a^2, a^3, a^4\}$ are 4 unique nonidentity elements that satisfy $x^5 = e$. \\
Now suppose that there exists $b \in G$ such that $b^5 = e$, $b \neq e$, and $b \notin \{a, a^2, a^3, a^4\}$. We will show that $\{b, b^2, b^3, b^4\}$ and $\{a, a^2, a^3, a^4\}$ are disjoint. \\
Suppose FSOC that $b^4 = a^i$ for some $i \in \{1, 2, 3, 4\}$. Then $e = a^ib \implies a^{5-i}=a^{5-i}a^ib \implies a^{5-i}=b$, which is a contradiction, so $b^4 \neq a^i$ for all $i \in \{1, 2, 3, 4\}$ \\
Suppose FSOC that $b^2 = a^i$ for some $i \in \{1, 2, 3, 4\}$. Then $(b^2)^2 = a^{2i} \implies b^4 = a^{2i}$, which contradicts the previous statement, so $b^2 \neq a^i$ for all $i \in \{1, 2, 3, 4\}$ \\
Suppose FSOC that $b^3 = a^i$ for some $i \in \{1, 2, 3, 4\}$. Then $e = a^ib^2 \implies a^{5-i} = b^2$, which contradicts the previous statement, so $b^3 \neq a^i$ for all $i \in \{1, 2, 3, 4\}$ \\
So $\{b, b^2, b^3, b^4\}$ and $\{a, a^2, a^3, a^4\}$ are disjoint, meaning that any $b \notin \{a, a^2, a^3, a^4\}$ will contribute 4 additional distinct solutions; since the group has finitely many elements, the total number of solutions is finite and a multiple of 4, as desired. \\
If the group is not finite (i.e. is infinite), the group could have infinitely such nonidentity elements that satisfy the equation $x^5 = e$.

\end{proof}

\subsection*{Problem 7 (Chapter 2, Exercise 52)}
Suppose that in the definition of a group $G$, the condition that for each element $a$ in $G$ there exists an element $b$ in $G$ with the property that $ab = ba = e$ is replaced by the condition that $ab = e$. Show that $ba = e$.

\begin{proof}

Let $a \in G$ be arbitrary. By assumption, there exists $b \in G$ such that $ab = e$. Left multiplying this expression by $b$ yields $bab = b$. Right cancellation of $b$ yields $ba = e$, which was to be shown.

\end{proof}

\subsection*{Problem 8 (Chapter 3, Exercise 4)}
Prove that in any group, an element and its inverse have the same order.

\begin{proof}

Let $a \in G$ be arbitrary with the property that $|a| = n$; that is, that $a^n = e$. Then $e = (aa^{-1})^n = a^n(a^{-1})^n = e(a^{-1})^n = (a^{-1})^n$, so by definition $|a^{-1}| = n$; interchanging the roles of $a$ and $a^{-1}$ proves the reverse implication.

\end{proof}

\subsection*{Problem 9 (Chapter 3, Exercise 14)}
Prove that if $a$ is the only element of order 2 in a group, then $a$ lies in the center of the group.

\begin{proof}

Suppose that $a \in G$ is the unique element of order 2; that is, that it is the only element such that $a^2 = e$. We deduce that $a = a^{-1}$. We want to show that for all $g \in G$ it follows that $ag = ga$. To this end, let $g \in G$ be arbitrary and consider $b = gag^{-1}$. Squaring both sides yields $b^2 = gag^{-1}gag^{-1} = gaag^{-1} = gaa^{-1}g^{-1} = gg^{-1} = e$. Since $a$ is the only element of order 2, we deduce that $b = a$, so $a = gag^{-1}$; right multiplying both sides by $g$ yields $ag = ga$, which was to be shown.

\end{proof}

\subsection*{Problem 10 (Chapter 3, Exercise 18)}
Suppose that $a$ is a group element and $a^6 = e$. What are the possibilities for $|a|$? Provide reasons for your answer.

\begin{proof}

Because $a^6 = e$, it follows that $|a| \leq 6$ by definition of order. \\
Suppose that $|a| = 1$, then $a = e$, meaning $a^6 = e^6 = e$. Thus, $|a| = 1$ is a possibility. \\ Suppose that $|a| = 2$, then $a^2 = e$, meaning $a^6 = (a^2)^3 = e^3 = e$. Thus, $|a| = 2$ is a possibility. \\
Suppose that $|a| = 3$, then $a^3 = e$, meaning $a^6 = (a^3)^2 = e^2 = e$. Thus, $|a| = 3$ is a possibility. \\
Suppose that $|a| = 4$, then $a^4 = e$, meaning $a^6 = e = a^4a^2 = ea^2$, implying that $a^2 = e$, which contradicts the fact that $|a| = 4$. Thus, $|a| = 4$ is not a possibility. \\
Suppose that $|a| = 5$, then $a^5 = e$, meaning $a^6 = e = a^5a = ea$, implying that $a = e$, which contradicts the fact that $|a| = 5$. Thus, $|a| = 5$ is not a possibility. \\
Suppose that $|a| = 6$, then $a^6 = e$. Thus, $|a| = 6$ is a possibility. \\

In summary, the possibilities of $|a|$ are 1, 2, 3, and 6—namely the divisors of 6.
\end{proof}

\end{document}
