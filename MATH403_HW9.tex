\documentclass{article}
\usepackage[utf8]{inputenc}
\usepackage[english]{babel}
\usepackage[]{amsthm} 
\usepackage{enumitem}
\usepackage{array}
\usepackage{amsmath}
\usepackage[]{amssymb} 
\usepackage{gensymb}
\DeclareMathOperator{\lcm}{lcm}

\title{HW 8 - MATH403}
\author{Danesh Sivakumar}
\date{April 4th, 2022}


\begin{document}
\maketitle 

\subsection*{Problem 1 (Chapter 24, Exercise 10)}
Let $H$ be a proper subgroup of a finite group $G$. Show that $G$ is not the union of all conjugates of $H$.

\begin{proof}

\end{proof}


\subsection*{Problem 2 (Chapter 24, Exercise 16)}
Find all the Sylow 3-subgroups of $S_4$.
\begin{proof}

\end{proof}

\subsection*{Problem 3 (Chapter 24, Exercise 22)}
Show that every group of order 56 has a proper nontrivial normal subgroup.
\begin{proof}

\end{proof}

\subsection*{Problem 4 (Chapter 24, Exercise 26)}
How many Sylow 5-subgroups of $S_5$ are there? Exhibit two.
\begin{proof} 

\end{proof}

\subsection*{Problem 5 (Chapter 24, Exercise 40)}
Suppose that $G$ is a group of order 60 and $G$ has a normal subgroup $N$ of order 2. Show that
\begin{enumerate}[label=\alph*.]
    \item $G$ has normal subgroups of orders 6, 10, and 30.
    \item $G$ has subgroups of orders 12 and 20.
    \item $G$ has a cyclic subgroup of order 30.
\end{enumerate}
\begin{proof} 

\end{proof}


\subsection*{Problem 6 (Chapter 24, Exercise 60)}
Determine the groups of order 45.
\begin{proof}

\end{proof}

\subsection*{Problem 7 (Chapter 12, Exercise 26)}
Determine $U(\mathbb{R}[x])$.
\begin{proof}

\end{proof}

\subsection*{Problem 8 (Chapter 12, Exercise 40)}
Let $M_2(\mathbb{Z})$ be the ring of all $2 \times 2$ matrices over the integers and let $R = \bigg \{ \begin{bmatrix} a & a + b \\ a + b & b \end{bmatrix} \biggm\vert a, b \in \mathbb{Z} \bigg \}$. Prove or disprove that $R$ is a subring of $M_2(\mathbb{Z})$.
\begin{proof}

\end{proof}

\subsection*{Problem 9 (Chapter 12, Exercise 54)}
Show that $4x^2 + 6x + 3$ is a unit in $\mathbb{Z}_8[x]$.
\begin{proof} 

\end{proof}

\subsection*{Problem 10}
Calculate the number of different conjugacy classes in $S_5$ and write down a representative element for each conjugacy class.
\begin{proof}

\end{proof}

\subsection*{Problem 11}
Prove that the 3-cycles in $A_5$ do form a single conjugacy class but that the 5-cycles do not.
\begin{proof}

\end{proof}



\end{document}
