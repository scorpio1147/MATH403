\documentclass{article}
\usepackage[utf8]{inputenc}
\usepackage[english]{babel}
\usepackage[]{amsthm} 
\usepackage{enumitem}
\usepackage{array}
\usepackage{amsmath}
\usepackage[]{amssymb} 
\usepackage{gensymb}
\DeclareMathOperator{\lcm}{lcm}

\title{HW 6 - MATH403}
\author{Danesh Sivakumar}
\date\today


\begin{document}
\maketitle 

\subsection*{Problem 1 (Chapter 8, Exercise 20)}
Find a subgroup of $\mathbb{Z}_{12} \oplus \mathbb{Z}_{18}$ that is isomorphic to $\mathbb{Z}_9 \oplus \mathbb{Z}_4$.

\begin{proof}
Note that $\mathbb{Z}_9 \oplus \mathbb{Z}_4$ is cyclic and of order 36 because 9 and 4 are coprime; also note that $3 \in \mathbb{Z}_{12}$ has order 4 and $2 \in \mathbb{Z}_{18}$ has order 9, so that $(3, 2) \in \mathbb{Z}_{12} \oplus \mathbb{Z}_{18}$ has order $4 \cdot 9 = 36$. This immediately implies that the subgroup generated by this element $H = \langle (3, 2) \rangle$ is cyclic and of order 36, so it is isomorphic to $\mathbb{Z}_9 \oplus \mathbb{Z}_4$.
\end{proof}


\subsection*{Problem 2 (Chapter 8, Exercise 22)}
Determine the number of elements of order 15 and the number of cyclic subgroups of order 15 in $\mathbb{Z}_{30} \oplus \mathbb{Z}_{20}$.

\begin{proof}
Note that elements of order 15 have the form $(a, b)$ where lcm($|a|$, $|b|$) = 15, where $|a|$ divides 30 and $|b|$ divides 20. There are three ways to make this happen: $(15, 1)$, $(15, 5)$ and $(3, 5)$. This immediately implies that the number of elements of order 15 is $\varphi(15)\varphi(1) + \varphi(15)\varphi(5) + \varphi(3)\varphi(5) = 8\cdot 1 + 8\cdot 4 + 2\cdot 4 = 48$. \\
Note that each cyclic subgroup of order 15 has $\varphi(15) = 8$ generators; this implies that there are $48/8 = 6$ distinct cyclic subgroups of order 15, because picking any other generator from the same "group" of 8 generators would yield the same cyclic subgroup.
\end{proof}

\subsection*{Problem 3 (Chapter 8, Exercise 60)}
Give an example of an infinite non-Abelian group that has exactly six elements of finite order.
\begin{proof}
Note that $\mathbb{Z} \oplus S_3$ satisfies these properties. Because $\mathbb{Z}$ is infinite, the direct product is also infinite. $S_3$ is also non-Abelian; note that $(12)(23) = (21)(23) = (231)$ but $(23)(12) = (23)(21) = (213)$ and $(231) \neq (231)$. Thus the direct product is non-Abelian. But there are exactly six elements of finite order in this group; the only element of finite order in $\mathbb{Z}$ is 0 and all 6 of the elements in $S_3$ have finite order; this means that all 6 of the elements of the form $(0, \sigma)$ are the only elements of finite order in this infinite non-Abelian group. 
\end{proof}

\subsection*{Problem 4 (Chapter 9, Exercise 10)}
Let $H = \{ (1), (12)(34)\}$ in $A_4$.
\begin{enumerate}[label=\alph*.]
    \item Show that $H$ is not normal in $A_4$.
    \item Referring to the multiplication table for $A_4$ in Table 5.1 on page 105, show that, although $\alpha_6H = \alpha_7H$ and $\alpha_9H = \alpha_{11}H$, it is not true that $\alpha_6\alpha_9H = \alpha_7\alpha_{11}H$. Explain why this proves that the left cosets of $H$ do not form a group under coset multiplication.
\end{enumerate}
\begin{proof} 
\qquad
\begin{enumerate}[label=\alph*.]
    \item Note that $(123) \in A_4$, but $(123)(12)(34)(123)^{-1} = (123)(12)(34)(321) = (13)(24) \notin H$.
    \item From the multiplication table, we observe that $\alpha_6H = \{(243), (142)\} = \alpha_7H$ and $\alpha_9H = \{(132), (234)\} = \alpha_{11}H$. However, we also note that $\alpha_6\alpha_9H = (243)(132)H = (12)(34)H = H$, but $\alpha_7\alpha_{11}H = (142)(234)H = (14)(23)H \neq H$, so that $\alpha_6\alpha_9H \neq \alpha_7\alpha_{11}H$. This shows that multiplication is not well-defined for the left cosets, meaning that the group operation fails to work; thus the left cosets of $H$ do not form a group under coset multiplication. 
\end{enumerate}
\end{proof}

\subsection*{Problem 5 (Chapter 9, Exercise 22)}
Determine the order of $(\mathbb{Z} \oplus \mathbb{Z})/\langle(2, 2)\rangle$. Is the group cyclic?
\begin{proof} 
Note that $(1, 0) + \langle (2, 2) \rangle$ is never in the form $(2k, 2k)$ for some $k \in \mathbb{Z}$, so it has infinite order; thus the group has infinite order. In order for it to be cyclic, it must be isomorphic to $\mathbb{Z}$, but this is not the case because $(1, 1) + \langle (2, 2) \rangle$ generates elements of order 2 (adding any of these elements to itself twice yields an element in the form $(2k, 2k)$), but $\mathbb{Z}$ has no elements of order 2; thus the group is not cyclic.
\end{proof}


\subsection*{Problem 6 (Chapter 9, Exercise 26)}
Let $H = \{1, 17, 41, 49, 73, 89, 97, 113\}$ under multiplication modulo 120. Write $H$ as an external direct product of groups of the form $\mathbb{Z}_{2^k}$. Write $H$ as an internal direct product of nontrivial subgroups. 
\begin{proof}
Notice that $H$ has 8 elements, meaning that it is isomorphic to $\mathbb{Z}_8$, $\mathbb{Z}_4 \oplus \mathbb{Z}_2$, or $\mathbb{Z}_2 \oplus \mathbb{Z}_2 \oplus \mathbb{Z}_2$. Note further that 1 is the only element of order 1, $\{17, 73, 97, 113\}$ are the elements of order 4 and $\{41, 49, 89\}$ are the elements of order 2. Because no elements have order 8 and there are elements of order 4, it follows that $H$ is isomorphic to $\mathbb{Z}_4 \oplus \mathbb{Z}_2$ as an external direct product. To express $H$ as an internal direct product, take any elements of order 4 and 2, say $\langle 17 \rangle \times \langle 41 \rangle$. Clearly the intersection of both of these subgroups is trivial and these subgroups are normal in $H$ because multiplication modulo n is commutative, so this internal direct product works; it is also isomorphic to our choice of external direct product.
\end{proof}

\subsection*{Problem 7 (Chapter 9, Exercise 38)}
Prove that for every positive integer $n$, $\mathbb{Q}/\mathbb{Z}$ has an element of order $n$.
\begin{proof}
We want to find a coset $\frac{p}{q}\mathbb{Z}$ such that $n\frac{p}{q}\mathbb{Z} = \mathbb{Z}$ and it is the smallest positive solution to this relation; clearly taking $\frac{1}{n}\mathbb{Z}$ works, and this holds for any positive integer.
\end{proof}

\subsection*{Problem 8 (Chapter 9, Exercise 48)}
If $G$ is a group and $|G\colon Z(G)| = 4$, prove that $G/Z(G) \cong \mathbb{Z}_2 \oplus \mathbb{Z}_2$.
\begin{proof}
Note that $G$ could possibly be isomorphic to $\mathbb{Z}_2 \oplus \mathbb{Z}_2$ or $\mathbb{Z}_4$. Assume it is isomorphic to $\mathbb{Z}_4$; then by the $G/Z$ theorem, $G$ is Abelian because $\mathbb{Z}_4$ is cyclic. But this implies that $|G\colon Z(G)| = |G| / |Z(G)| = |G| / |G| = 1$, which is a contradiction; thus $G/Z(G)$ must be isomorphic to $\mathbb{Z}_2 \oplus \mathbb{Z}_2$.
\end{proof}

\subsection*{Problem 9 (Chapter 9, Exercise 52)}
Let $G$ be an Abelian group and let $H$ be the subgroup consisting of all elements of $G$ that have finite order. Prove that every nonidentity element in $G/H$ has infinite order.
\begin{proof} 
Pick a nonidentity element $gH \in G/H$ such that $gH \neq H$; that is $g$ has infinite order. Suppose that $gH$ has finite order $n$; then $H = (gH)^n = (g^n)H$, which implies that $g^n$ has finite order; but this implies that $g$ also has finite order, which contradicts our assumption that $gH \neq H$; thus every nonidentity element has infinite order.
\end{proof}

\subsection*{Problem 10 (Chapter 9, Exercise 58)}
If $N$ and $M$ are normal subgroups of $G$, prove that $NM$ is also a normal subgroup of $G$.
\begin{proof}
First, we prove that $NM$ is a subgroup of $G$ with the one-step subgroup test. Clearly $NM$ is nonempty, because $ee = e \in NM$. Take $n_1m_1, n_2m_2, \in NM$; then $(n_1m_1)(n_2m_2)^{-1} = n_1m_1m_2^{-1}n_2^{-1}$. The normality of $M$ implies that there exists $m_3 \in M$ such that $m_3 = n_2m_1m_2^{-1}n_2^{-1}$; this means that $n_1m_1m_2^{-1}n_2^{-1} = n_1n_2^{-1}m_3 \in NM$, so that $NM \leq G$. Because $N$ and $M$ are normal in $G$, it follows that for any $g \in G$ that $gNg^{-1} \subset N$ and $gMg^{-1} \subset M$; thus $gNMg^{-1} = gNg^{-1}gMg^{-1} \subset NM$; thus $NM$ is a normal subgroup of $G$.
\end{proof}

\subsection*{Problem 11 (Chapter 9, Exercise 72)}
Let $G$ be a group and $H$ an odd-order subgroup of $G$ of index 2. Show that $H$ contains every element of $G$ of odd order.
\begin{proof}
Suppose $g \in G$ has odd order; then $g^2$ is a generator of $\langle g \rangle$ because 2 is coprime to any odd integer. But the fact that $H$ is of index 2 implies that the order of $G/H$ is 2, meaning $g^2H = H$. This means $g^2 \in H$, so that $g \in \langle g^2 \rangle \leq H$ must also be contained in $H$.
\end{proof}


\end{document}
